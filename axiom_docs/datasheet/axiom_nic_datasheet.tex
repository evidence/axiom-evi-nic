\documentclass[10pt,a4paper]{paper}
\usepackage[latin1]{inputenc}
\usepackage[margin=2cm]{geometry}
\usepackage{amsmath}
\usepackage{amsfonts}
\usepackage{amssymb}
\usepackage{graphicx}
\usepackage{tabularx}
\usepackage{environ}
\usepackage[colorlinks,linkcolor=blue]{hyperref}

\NewEnviron{regdescription}
	{\begin{tabbing}
	\hspace{150pt}\=\kill
		\BODY
	\end{tabbing}}

\NewEnviron{regglobalsummary}
	{\begin{tabularx}{\textwidth}{|l|l|l|l|l|X|}
	        \hline \textbf{Module} & \textbf{Register Name} & \textbf{Address} & \textbf{Width} & \textbf{Type} &  \textbf{Description} \\
		\BODY
	\hline
	\end{tabularx}}

\NewEnviron{regsummary}
	{\begin{tabularx}{\textwidth}{|l|l|l|l|l|X|}
	        \hline \textbf{Register Name} & \textbf{Address} & \textbf{Width} & \textbf{Type} & \textbf{Init Val} & \textbf{Description} \\
		\BODY
	\hline
	\end{tabularx}}

\NewEnviron{regdetails}
	{\begin{tabularx}{\textwidth}{|l|l|l|l|X|}
	        \hline \textbf{Field Name} & \textbf{Bit(s)} & \textbf{Type} & \textbf{Init Val} & \textbf{Description} \\
	 	\BODY
	\hline 
	\end{tabularx}}

% Title Page
\title{Axiom NIC Datasheet}
\subtitle{v0.1 - \today}
\author{Evidence SRL}

\begin{document}
\maketitle


%\chapter{Axiom NIC}
\section{AXIOM NIC Registers}
\begin{regdescription}
	Total size      \> 5136 bytes\\
\end{regdescription}

\subsection{Register summary}
\begin{regglobalsummary}
        \hline \textbf{\nameref{mod:status}} & & & & & \\
                                             & & & & & \\
        \hline \nameref{mod:status} & \nameref{reg:version} & 0x0000 & 32 & ro & Version register\\
	\hline \nameref{mod:status} & \nameref{reg:status} & 0x0004 & 32 & ro & Status register\\
	\hline \nameref{mod:status} & \nameref{reg:ifnumber} & 0x0008 & 32 & ro & Number of interface\\
        \hline \nameref{mod:status} & \nameref{reg:ifinfo_base} & 0x000C+n, & 8 & ro & Interface status table\\
                               & n=0...3 & & & & \\
        \hline \textbf{\nameref{mod:control}} & & & & & \\
                                              & & & & & \\
	\hline \nameref{mod:control} & \nameref{reg:control} & 0x0040 & 32 & rw & Control register \\
	\hline \nameref{mod:control} & \nameref{reg:nodeid} & 0x0044 & 32 & rw & Node ID register \\
        \hline \textbf{\nameref{mod:interrupt}} & & & & & \\
                                                & & & & & \\
	\hline \nameref{mod:interrupt} & \nameref{reg:ackirq} & 0x0060 & 32 & wo & ACK Interrupt Register\\
	\hline \nameref{mod:interrupt} & \nameref{reg:setirq} & 0x0064 & 32 & rw & Set Interrupt Register\\
	\hline \nameref{mod:interrupt} & \nameref{reg:pndirq} & 0x0068 & 32 & ro & Pending Interrupt Register\\
        \hline \textbf{\nameref{mod:routing}} & & & & & \\
                                              & & & & & \\
        \hline \nameref{mod:routing} & \nameref{reg:routing_base} & 0x0080+n & 8 & rw & Routing table base register\\
                            & n=0...255 & & & & \\
        \hline \textbf{\nameref{mod:queue}} & & & & & \\
                                            & & & & & \\
        \hline \nameref{mod:queue} & \nameref{reg:raw_tx_head} & 0x0200 & 32 & ro & RAW TX queue head\\
	\hline \nameref{mod:queue} & \nameref{reg:raw_tx_tail} & 0x0204 & 32 & ro & RAW TX queue tail\\
	\hline \nameref{mod:queue} & \nameref{reg:raw_tx_info} & 0x0208 & 32 & ro & RAW TX queue info\\
	\hline \nameref{mod:queue} & \nameref{reg:raw_tx_start} & 0x020C & 32 & wo & RAW TX queue start\\
        \hline \nameref{mod:queue} & \nameref{reg:raw_tx_base} & 0x0210+8*n & 64 & wo & RAW TX queue base register\\
                                & n=0...255 & & & & \\
	\hline \nameref{mod:queue} & \nameref{reg:raw_rx_head} & 0x0C00 & 32 & ro & RAW TX queue head\\
	\hline \nameref{mod:queue} & \nameref{reg:raw_rx_tail} & 0x0C04 & 32 & ro & RAW TX queue tail\\
	\hline \nameref{mod:queue} & \nameref{reg:raw_rx_info} & 0x0C08 & 32 & ro & RAW TX queue info\\
	\hline \nameref{mod:queue} & \nameref{reg:raw_rx_start} & 0x0C0C & 32 & wo & RAW TX queue start\\
        \hline \nameref{mod:queue} & \nameref{reg:raw_rx_base} & 0x0C10+8*n & 64 & ro & RAW TX queue base register\\
                                & n=0...255 & & & & \\
\end{regglobalsummary}


\section{Status Registers} \label{mod:status}
\begin{regdescription}
	Module Name 	\> Status Registers\\
	Description 	\> Registers used to check the status of the device\\
\end{regdescription}

\subsection{Register summary}
\begin{regsummary}
	\hline \nameref{reg:version} & 0x0000 & 32 & ro & 0x0 & Version register\\
	\hline \nameref{reg:status} & 0x0004 & 32 & ro & 0x0 & Status register\\
	\hline \nameref{reg:ifnumber} & 0x0008 & 32 & ro & 0x0 & Number of interface\\
        \hline \nameref{reg:ifinfo_base} & 0x000C+n, & 8 & ro & 0x0 & Interface status table\\
                               & n=0...3 & & & & \\
\end{regsummary}


\subsubsection{VERSION} \label{reg:version}
\begin{regdescription}
	Name			\> VERSION\\
	Relative Address	\> 0x00000000\\
	Width			\> 32 bits\\
	Access Type		\> ro\\
	Init Value		\> 0x00000000\\
	Description		\> Version register\\
\end{regdescription}
\begin{regdetails}
	\hline reserved & 31:16 & ro & 0x0 & Reserved. Writes are ignored, read data is zero.\\
	\hline BITSTREAM & 15:8 & ro & 0x0 & Bitstream version\\
	\hline BOARD & 7:0 & ro & 0x0 & Axiom board model\\
\end{regdetails}


\subsubsection{STATUS} \label{reg:status}
\begin{regdescription}
	Name			\> STATUS\\
	Relative Address	\> 0x00000004\\
	Width			\> 32 bits\\
	Access Type		\> ro\\
	Init Value		\> 0x00000000\\
	Description		\> Status register\\
\end{regdescription}
\begin{regdetails}
	\hline reserved & 31:1 & ro & 0x0 & Reserved. Writes are ignored, read data is zero.\\
	\hline TBD & 0 & rw & 0x0 & TBD\\
\end{regdetails}


\subsubsection{IFNUMBER} \label{reg:ifnumber}
\begin{regdescription}
	Name			\> IFNUMBER\\
	Relative Address	\> 0x00000008\\
	Width			\> 32 bits\\
	Access Type		\> ro\\
	Init Value		\> 0x00\\
	Description		\> Number of interface\\
\end{regdescription}
\begin{regdetails}
	\hline IFNUMBER & 7:0 & ro & 0x2 & Number of interface which are present on a node.\\
\end{regdetails}


\subsubsection{IFINFO\_BASE[n]} \label{reg:ifinfo_base}
\begin{regdescription}
	Name			\> IFINFO\_BASE[n]\\
	Relative Address	\> 0x0000000C+n, n=0...3\\
	Width (single row)	\> 8 bits\\
	Access Type		\> ro\\
	Init Value		\> 0x00\\
	Description		\> Interface status table\\
\end{regdescription}
\begin{regdetails}
	\hline reserved & 7:3 & ro & 0x0 & Reserved. Writes are ignored, read data is zero.\\
	\hline CONNECTED & 2 & ro & 0x0 & Interface connected status.\\
               & & & & 1b = interface is physically connected to another board.\\
               & & & & 0b = interface is disconnected.\\
	\hline RX & 1 & ro & 0x0 & Interface RX functionality.\\
               & & & & This bit is set when the interface can work as RX.\\
	\hline TX & 0 & ro & 0x0 & Interface TX functionality.\\
               & & & & This bit is set when the interface can work as TX.\\
\end{regdetails}



\section{Control Registers} \label{mod:control}
\begin{regdescription}
	Module Name 	\> Control Registers\\
	Description 	\> Registers used to control the device\\
\end{regdescription}

\subsection{Register summary}
\begin{regsummary}
	\hline \nameref{reg:control} & 0x0040 & 32 & rw & 0x0 & Control register \\
	\hline \nameref{reg:nodeid} & 0x0044 & 32 & rw & 0x0 & Node ID register \\
\end{regsummary}


\subsubsection{CONTROL} \label{reg:control}
\begin{regdescription}
	Name			\> CONTROL\\
	Relative Address	\> 0x00000040\\
	Width			\> 32 bits\\
	Access Type		\> rw\\
	Init Value		\> 0x00000000\\
	Description		\> Control register. Used to enable local transmission ACKs\\
	                        \> and/or the PHY loopback mode configuration.\\
\end{regdescription}
\begin{regdetails}
	\hline reserved & 31:1 & rw & 0x0 & Reserved. Writes are ignored, read data is zero.\\
	\hline LOOPBACK & 0 & rw & 0x0 & When set to 1b, the interface is in loopback mode.\\
\end{regdetails}


\subsubsection{NODEID} \label{reg:nodeid}
\begin{regdescription}
	Name			\> NODEID\\
	Relative Address	\> 0x00000044\\
	Width			\> 32 bits\\
	Access Type		\> rw\\
	Init Value		\> 0x00\\
	Description		\> Node ID register\\
\end{regdescription}
\begin{regdetails}
	\hline NODEID & 7:0 & rw & 0x0 & Defines the ID of a local node.\\
\end{regdetails}



\section{Interrupt Registers} \label{mod:interrupt}
\begin{regdescription}
	Module Name 	\> Interrupt Registers\\
	Description 	\> Registers used to handle interrupts\\
\end{regdescription}

\subsection{Register summary}
\begin{regsummary}
	\hline \nameref{reg:ackirq} & 0x0060 & 32 & wo & 0x0 & ACK Interrupt Register\\
	\hline \nameref{reg:setirq} & 0x0064 & 32 & rw & 0x0 & Set Interrupt Register\\
	\hline \nameref{reg:pndirq} & 0x0068 & 32 & ro & 0x0 & Pending Interrupt Register\\
\end{regsummary}

\subsubsection{ACKIRQ} \label{reg:ackirq}
\begin{regdescription}
	Name			\> ACKIRQ\\
	Relative Address	\> 0x00000060\\
	Width			\> 32 bits\\
	Access Type		\> wo\\
	Init Value		\> 0x00000000\\
	Description		\> ACK Interrupt Register. Used to send an ack for the interrupts\\
\end{regdescription}
\begin{regdetails}
	\hline reserved & 31:3 & wo & 0x0 & Reserved. Writes are ignored.\\
	\hline IRQ\_RAW\_TX & 2 & wo & 0x0 & Interrupt RAW TX queue.\\
               & & & & When set to 1b, the RAW TX interrupt is acknowledged.\\
	\hline IRQ\_RAW\_RX & 1 & wo & 0x0 & Interrupt RAW RX queue.\\
               & & & & When set to 1b, the RAW RX interrupt is acknowledged.\\
	\hline IRQ\_GENERIC & 0 & wo & 0x0 & Interrupt generic.\\
               & & & & When set to 1b, the GENERIC interrupt is acknowledged.\\
\end{regdetails}

\subsubsection{SETIRQ} \label{reg:setirq}
\begin{regdescription}
	Name			\> SETIRQ\\
	Relative Address	\> 0x00000064\\
	Width			\> 32 bits\\
	Access Type		\> rw\\
	Init Value		\> 0x00000000\\
	Description		\> Set Interrupt Register. Used to enable/disable interrupts\\
\end{regdescription}
\begin{regdetails}
	\hline reserved & 31:3 & rw & 0x0 & Reserved. Writes are ignored, read data is zero.\\
	\hline IRQ\_RAW\_TX & 2 & rw & 0x0 & Interrupt RAW TX queue.\\
               & & & & 1b = RAW TX interrupt is enabled.\\
               & & & & 0b = RAW TX interrupt is disabled.\\
	\hline IRQ\_RAW\_RX & 1 & rw & 0x0 & Interrupt RAW RX queue.\\
               & & & & 1b = RAW RX interrupt is enabled.\\
               & & & & 0b = RAW RX interrupt is disabled.\\
	\hline IRQ\_GENERIC & 0 & rw & 0x0 & Interrupt generic.\\
               & & & & 1b = GENERIC interrupt is enabled.\\
               & & & & 0b = GENERIC interrupt is disabled.\\
\end{regdetails}

\subsubsection{PNDIRQ} \label{reg:pndirq}
\begin{regdescription}
	Name			\> PNDIRQ\\
	Relative Address	\> 0x00000068\\
	Width			\> 32 bits\\
	Access Type		\> ro\\
	Init Value		\> 0x00000000\\
	Description		\> Pending Interrupt Register. Used to report the cause of the interrupt\\
\end{regdescription}
\begin{regdetails}
	\hline reserved & 31:3 & ro & 0x0 & Reserved. Read data is zero.\\
	\hline IRQ\_RAW\_TX & 2 & ro & 0x0 & Interrupt RAW TX queue.\\
               & & & & This bit is set to 1b when the RAW TX queue generates an interrupt.\\
               & & & & This bit is reset to 0b when the RAW TX queue interrupt is acknowledged.\\
	\hline IRQ\_RAW\_RX & 1 & ro & 0x0 & Interrupt RAW RX queue.\\
               & & & & This bit is set to 1b when the RAW RX queue generates an interrupt.\\
               & & & & This bit is reset to 0b when the RAW RX queue interrupt is acknowledged.\\
	\hline IRQ\_GENERIC & 0 & ro & 0x0 & Interrupt generic.\\
               & & & & This bit is set to 1b when the GENERIC interrupt is raised.\\
               & & & & This bit is reset to 0b when the GENERIC interrupt is acknowledged.\\
\end{regdetails}



\section{Routing Registers} \label{mod:routing}
\begin{regdescription}
	Module Name 	\> Routing Registers\\
	Description 	\> Registers used to manage the routing table\\
\end{regdescription}

\subsection{Register summary}
\begin{regsummary}
    \hline \nameref{reg:routing_base} & 0x0080+n & 8 & rw & 0x00 & Routing table base register\\
                            & n=0...255 & & & & \\
\end{regsummary}

\subsubsection{ROUTING\_BASE[n]} \label{reg:routing_base}
\begin{regdescription}
	Name			\> RAW\_RX\_BASE[n]\\
	Relative Address	\> 0x00000080+n, n=0...255\\
	Width (single row)	\> 8 bits\\
	Access Type		\> rw\\
	Init Value		\> 0x00\\
	Description		\> Routing table base register\\
	                        \> This array of registers represents the routing table.\\
	                        \> The array index represents the destination node id.\\
\end{regdescription}
\begin{regdetails}
	\hline reserved & 7:4 & rw & 0x0 & Reserved. Writes are ignored, read data is zero.\\
        \hline ENABLED\_IF\_3 & 3 & rw & 0x0 & When set to 1b, the node n (array index) is reachable through the interface 3.\\
        \hline ENABLED\_IF\_2 & 2 & rw & 0x0 & When set to 1b, the node n (array index) is reachable through the interface 2.\\
        \hline ENABLED\_IF\_1 & 1 & rw & 0x0 & When set to 1b, the node n (array index) is reachable through the interface 1.\\
        \hline ENABLED\_IF\_0 & 0 & rw & 0x0 & When set to 1b, the node n (array index) is reachable through the interface 0.\\
\end{regdetails}



\subsection{Queues Registers} \label{mod:queue}
\begin{regdescription}
	Module Name 	\> Queues Registers\\
	Description 	\> Registers used to manage the descriptor queues\\
\end{regdescription}

\subsection{Register summary}
\begin{regsummary}
        \hline \nameref{reg:raw_tx_head} & 0x0200 & 32 & ro & 0x0 & RAW TX queue head\\
	\hline \nameref{reg:raw_tx_tail} & 0x0204 & 32 & ro & 0x0 & RAW TX queue tail\\
	\hline \nameref{reg:raw_tx_info} & 0x0208 & 32 & ro & 0x0 & RAW TX queue info\\
	\hline \nameref{reg:raw_tx_start} & 0x020C & 32 & wo & 0x0 & RAW TX queue start\\
        \hline \nameref{reg:raw_tx_base} & 0x0210+8*n & 64 & wo & 0x0 & RAW TX queue base register\\
                                & n=0...255 & & & & \\
	\hline \nameref{reg:raw_rx_head} & 0x0C00 & 32 & ro & 0x0 & RAW TX queue head\\
	\hline \nameref{reg:raw_rx_tail} & 0x0C04 & 32 & ro & 0x0 & RAW TX queue tail\\
	\hline \nameref{reg:raw_rx_info} & 0x0C08 & 32 & ro & 0x0 & RAW TX queue info\\
	\hline \nameref{reg:raw_rx_start} & 0x0C0C & 32 & wo & 0x0 & RAW TX queue start\\
        \hline \nameref{reg:raw_rx_base} & 0x0C10+8*n & 64 & ro & 0x0 & RAW TX queue base register\\
                                & n=0...255 & & & & \\
\end{regsummary}

\subsubsection{RAW\_TX\_HEAD} \label{reg:raw_tx_head}
\begin{regdescription}
	Name			\> RAW\_TX\_HEAD\\
	Relative Address	\> 0x00000200\\
	Width			\> 32 bits\\
	Access Type		\> ro\\
	Init Value		\> 0x00\\
	Description		\> RAW TX queue head\\
\end{regdescription}
\begin{regdetails}
	\hline RAW\_TX\_HEAD & 7:0 & ro & 0x0 & RAW TX queue head.\\
               & & & &  This register contains the head pointer for the RAW TX queue.
                        The head (in the TX queue) points to the first descriptor to be sent by
                        the network card. Hardware controls this pointer. When the descriptor
                        pointed by the head is sent, this register is incremented.\\
\end{regdetails}

\subsubsection{RAW\_TX\_TAIL} \label{reg:raw_tx_tail}
\begin{regdescription}
	Name			\> RAW\_TX\_TAIL\\
	Relative Address	\> 0x00000204\\
	Width			\> 32 bits\\
	Access Type		\> ro\\
	Init Value		\> 0x00\\
	Description		\> RAW TX queue tail\\
\end{regdescription}
\begin{regdetails}
	\hline RAW\_TX\_TAIL & 7:0 & ro & 0x0 & RAW TX queue tail.\\
               & & & &  This register contains the tail pointer for the RAW TX queue.
                        The tail (in the TX queue) points to the first empty descriptor
                        which the software can fill. Hardware controls this pointer. When the
                        software writes on \nameref{reg:raw_tx_start}, this register
                        is incremented.\\
\end{regdetails}

\subsubsection{RAW\_TX\_INFO} \label{reg:raw_tx_info}
\begin{regdescription}
	Name			\> RAW\_TX\_INFO\\
	Relative Address	\> 0x00000208\\
	Width			\> 32 bits\\
	Access Type		\> ro\\
	Init Value		\> 0x00\\
	Description		\> RAW TX queue info\\
\end{regdescription}
\begin{regdetails}
	\hline reserved & 7:2 & ro & 0x0 & Reserved. Read data is zero.\\
	\hline RAW\_STATUS\_FULL & 1 & ro & 0x0 & RAW TX queue full.\\
               & & & & This bit is set to 1b when the RAW TX queue is full.\\
	\hline RAW\_STATUS\_EMPTY & 0 & ro & 0x0 & RAW TX queue empty.\\
               & & & & This bit is set to 1b when the RAW TX queue is empty.\\
\end{regdetails}

\subsubsection{RAW\_TX\_START} \label{reg:raw_tx_start}
\begin{regdescription}
	Name			\> RAW\_TX\_START\\
	Relative Address	\> 0x0000020C\\
	Width			\> 32 bits\\
	Access Type		\> wo\\
	Init Value		\> 0x00\\
	Description		\> RAW TX queue start\\
\end{regdescription}
\begin{regdetails}
	\hline RAW\_TX\_START & 7:0 & ro & 0x0 & RAW TX queue start.\\
                       & & & & When set to value X, the \nameref{reg:raw_tx_tail} is
                                incremented by X (the hardware prevents the head overtaking).
                                After that, the new X descriptors are ready
                                to be sent by the hardware. The software should fill the
                                descriptors pointed by \nameref{reg:raw_tx_tail},
                                before setting this registers.\\
\end{regdetails}

\subsubsection{RAW\_TX\_BASE[n]} \label{reg:raw_tx_base}
\begin{regdescription}
	Name			\> RAW\_TX\_BASE[n]\\
	Relative Address	\> 0x00000210+8*n, n=0...255\\
	Width (single row)	\> 64 bits\\
	Access Type		\> wo\\
	Init Value		\> 0x00\\
	Description		\> RAW TX queue base register\\
	                        \> This array of registers represents the queue of RAW TX descriptors.\\
	                        \> Each descriptor is 64-bit.\\
\end{regdescription}
\begin{regdetails}
	\hline DATA & 63:32 & wo & 0x0 & Data to be sent. \\
	\hline TYPE & 31:24 & wo & 0x0 & RAW message type.\\
        \hline FLAGS & 23:16 & wo & 0x0 & RAW message flags. (NEIGHBOUR, OVERFLOW, etc.)\\
	\hline DST\_NODE& 15:8 & wo & 0x0 & Receiver node id.\\
	\hline SRC\_NODE& 7:0 & wo & 0x0 & Sender node id.\\
\end{regdetails}

\subsubsection{RAW\_RX\_HEAD} \label{reg:raw_rx_head}
\begin{regdescription}
	Name			\> RAW\_RX\_HEAD\\
	Relative Address	\> 0x00000C00\\
	Width			\> 32 bits\\
	Access Type		\> ro\\
	Init Value		\> 0x00\\
	Description		\> RAW RX queue head\\
\end{regdescription}
\begin{regdetails}
	\hline RAW\_RX\_HEAD & 7:0 & ro & 0x0 & RAW RX queue head.\\
               & & & &  This register contains the head pointer for the RAW RX queue.
                        The head (in the RX queue) points to the first descriptor received by
                        the network card, which the software can read.
                        Hardware controls this pointer. When the software writes on
                        \nameref{reg:raw_rx_start}, this register is incremented.\\
\end{regdetails}

\subsubsection{RAW\_RX\_TAIL} \label{reg:raw_rx_tail}
\begin{regdescription}
	Name			\> RAW\_RX\_TAIL\\
	Relative Address	\> 0x00000C04\\
	Width			\> 32 bits\\
	Access Type		\> ro\\
	Init Value		\> 0x00\\
	Description		\> RAW RX queue tail\\
\end{regdescription}
\begin{regdetails}
	\hline RAW\_RX\_TAIL & 7:0 & ro & 0x0 & RAW RX queue tail.\\
               & & & &  This register contains the tail pointer for the RAW RX queue.
                        The tail (in the RX queue) points to the first empty descriptor
                        which the hardware can fill. Hardware controls this pointer.
                        When the descriptor pointed by the tail is filled by the hardware,
                        this register is incremented.\\
\end{regdetails}

\subsubsection{RAW\_RX\_INFO} \label{reg:raw_rx_info}
\begin{regdescription}
	Name			\> RAW\_RX\_INFO\\
	Relative Address	\> 0x00000C08\\
	Width			\> 32 bits\\
	Access Type		\> ro\\
	Init Value		\> 0x00\\
	Description		\> RAW RX queue info\\
\end{regdescription}
\begin{regdetails}
	\hline reserved & 7:2 & ro & 0x0 & Reserved. Read data is zero.\\
	\hline RAW\_STATUS\_FULL & 1 & ro & 0x0 & RAW RX queue full.\\
               & & & & This bit is set to 1b when the RAW RX queue is full.\\
	\hline RAW\_STATUS\_EMPTY & 0 & ro & 0x0 & RAW RX queue empty.\\
               & & & & This bit is set to 1b when the RAW RX queue is empty.\\
\end{regdetails}

\subsubsection{RAW\_RX\_START} \label{reg:raw_rx_start}
\begin{regdescription}
	Name			\> RAW\_RX\_START\\
	Relative Address	\> 0x00000C0C\\
	Width			\> 32 bits\\
	Access Type		\> wo\\
	Init Value		\> 0x00\\
	Description		\> RAW RX queue start\\
\end{regdescription}
\begin{regdetails}
	\hline RAW\_RX\_START & 7:0 & ro & 0x0 & RAW RX queue start.\\
                       & & & & When set to value X, the \nameref{reg:raw_rx_head} is
                                incremented by X (the hardware prevents the tail overtaking).
                                After that, the new X descriptors are consumed by the
                                software and ready to be reused by the hardware.
                                The software should read the
                                descriptors pointed by \nameref{reg:raw_rx_head},
                                before setting this register.\\
\end{regdetails}

\subsubsection{RAW\_RX\_BASE[n]} \label{reg:raw_rx_base}
\begin{regdescription}
	Name			\> RAW\_RX\_BASE[n]\\
	Relative Address	\> 0x00000C10+8*n, n=0...255\\
	Width (single row)	\> 64 bits\\
	Access Type		\> ro\\
	Init Value		\> 0x00\\
	Description		\> RAW RX queue base register\\
	                        \> This array of registers represents the queue of RAW RX descriptors.\\
	                        \> Each descriptor is 64-bit.\\
\end{regdescription}
\begin{regdetails}
	\hline DATA & 63:32 & ro & 0x0 & Data received. \\
	\hline TYPE & 31:24 & ro & 0x0 & RAW message type.\\
        \hline FLAGS & 23:16 & ro & 0x0 & RAW message flags. (NEIGHBOUR, OVERFLOW, etc.)\\
	\hline DST\_NODE& 15:8 & ro & 0x0 & Receiver node id.\\
	\hline SRC\_NODE& 7:0 & ro & 0x0 & Sender node id.\\
\end{regdetails}

\end{document}
