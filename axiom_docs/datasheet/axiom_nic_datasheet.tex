\documentclass[10pt,a4paper]{paper}
\usepackage[latin1]{inputenc}
\usepackage[margin=2cm]{geometry}
\usepackage{amsmath}
\usepackage{amsfonts}
\usepackage{amssymb}
\usepackage{graphicx}
\usepackage{tabularx}
\usepackage{environ}
\usepackage{textcomp}
\usepackage[colorlinks,linkcolor=blue]{hyperref}

\NewEnviron{regdescription}
	{\begin{tabbing}
	\hspace{150pt}\=\kill
		\BODY
	\end{tabbing}}

\NewEnviron{regglobalsummary}
	{\begin{tabularx}{\textwidth}{|l|l|l|l|X|}
	        \hline \textbf{Register Name} & \textbf{Address} &
	        \textbf{Size} & \textbf{Type} & \textbf{Description} \\
		\BODY
	\hline
	\end{tabularx}}

\NewEnviron{regsummary}
	{\begin{tabularx}{\textwidth}{|l|l|l|l|l|X|}
	        \hline \textbf{Register Name} & \textbf{Address} &
	        \textbf{Size} & \textbf{Type} & \textbf{Init Val} &
	        \textbf{Description} \\
		\BODY
	\hline
	\end{tabularx}}

\NewEnviron{regdetails}
	{\begin{tabularx}{\textwidth}{|l|l|l|l|X|}
	        \hline \textbf{Field Name} & \textbf{Bit(s)} & \textbf{Type} &
	        \textbf{Init Val} & \textbf{Description} \\
	 	\BODY
	\hline
	\end{tabularx}}

\newcommand{\versionapi}{v0.14}

% Title Page
\title{Axiom NIC Datasheet}
\subtitle{\versionapi - \today}
\author{Evidence SRL - FORTH}

\begin{document}
\maketitle


%\chapter{Axiom NIC}
\section{AXIOM NIC Registers}

\subsection{Register summary}
\begin{regglobalsummary}
        \hline \textbf{\nameref{mod:status}} & & & & \\
                                             & & & & \\
        \hline \nameref{reg:version} & 0xA0170000 & 32 & ro & Version register\\
	\hline \nameref{reg:ifnumber} & 0xA0140000 & 32 & ro & Number of
	interface\\
        \hline \nameref{reg:ifinfo_base_1} & 0xA0150000 & 8 & ro & Interface
        status table\\
        \hline \nameref{reg:ifinfo_base_2} & 0xA0160000 & 8 & ro & Interface
        status table\\
        \hline \textbf{\nameref{mod:control}} & & & & \\
                                              & & & & \\
	\hline \nameref{reg:nodeid} & 0xA00A0000 & 32 & rw & Node ID register \\
	\hline \nameref{reg:dma_start} & 0xA0080000 & 64 & wo & DMA start
	register\\
	\hline \nameref{reg:dma_end} & 0xA0070000 & 64 & wo & DMA end
	register \\
        \hline \textbf{\nameref{mod:interrupt}} & & & & \\
                                                & & & & \\
	\hline \nameref{reg:mskirq} & 0xA0090000 & 32 & rw & Mask Interrupt
	Register\\
	\hline \nameref{reg:pndirq} & 0xA00B0000 & 32 & rw & Pending Interrupt
	Register\\
        \hline \textbf{\nameref{mod:routing}} & & & & \\
                                              & & & & \\
        \hline \nameref{reg:PHY1_routing_base} & 0xA0010000+n & 8 & rw & Routing
        table base register\\ & n=0...255 & & & \\
        \hline \nameref{reg:rx_routing_base} & 0xA0014000+n & 8 & rw & Routing
        table base register\\ & n=0...255 & & & \\
        \hline \nameref{reg:tx_dma_routing_base} & 0xA0016000+n & 8 & rw & Routing
        table base register\\ & n=0...255 & & & \\
        \hline \nameref{reg:tx_raw_routing_base} & 0xA0018000+n & 8 & rw & Routing
        table base register\\ & n=0...255 & & & \\
        \hline \textbf{\nameref{mod:queue}} & & & & \\
                                            & & & & \\
	\hline \nameref{reg:raw_tx_status} & 0xA0120000+0x0c & 32 & ro & RAW TX queue
	status register\\
        \hline \nameref{reg:raw_tx_desc} & 0xA0130000 & 32 & wo & RAW TX queue
        descriptor register\\
	\hline \nameref{reg:raw_rx_status} & 0xA00E0000+0x1c & 32 & ro & RAW RX queue
	status register\\
        \hline \nameref{reg:raw_rx_desc} & 0xA00F0000+0x1000 & 32 & ro & RAW RX queue
        descriptor register\\
	\hline \nameref{reg:rdma_tx_status} & 0xA0100000+0x0c & 32 & ro & RDMA TX
	queue status register\\
        \hline \nameref{reg:rdma_tx_desc} & 0xA0110000 & 32 & wo & RDMA TX queue
        descriptor register\\
	\hline \nameref{reg:rdma_rx_status} & 0xA00C0000+0x1c & 32 & ro & RDMA RX
	queue status register\\
        \hline \nameref{reg:rdma_rx_desc} & 0xA00D0000+0x1000 & 32 & ro & RDMA RX queue
        descriptor register\\
        \hline \nameref{reg:long_buf_base} & 0xA0012000+8*n & 64 & rw & LONG
        buffers base register\\ & n=0...31 & & & \\
\end{regglobalsummary}


\section{Status Registers} \label{mod:status}
\begin{regdescription}
	Module Name 	\> Status Registers\\
	Description 	\> Registers used to check the status of the device\\
0x00000620\end{regdescription}

\subsection{Register summary}
\begin{regsummary}
	\hline \nameref{reg:version} & 0xA0170000 & 32 & ro & 0x0 & Version
	register\\
	\hline \nameref{reg:ifnumber} & 0xA0140000 & 32 & ro & 0x0 & Number of
	interface\\
        \hline \nameref{reg:ifinfo_base_1} & 0xA0150000 & 8 & ro & 0x0 & Interface
        status table\\
        \hline \nameref{reg:ifinfo_base_2} & 0xA0160000 & 8 & ro  & 0x0 & Interface
        status table\\
\end{regsummary}


\subsubsection{VERSION} \label{reg:version}
\begin{regdescription}
	Name			\> VERSION\\
	Relative Address	\> 0xA0170000\\
	Width			\> 32 bits\\
	Access Type		\> ro\\
	Init Value		\> 0x00000000\\
	Description		\> Version register\\
\end{regdescription}
\begin{regdetails}
	\hline reserved & 31:16 & ro & 0x0 & Reserved. Writes are ignored, read
	data is zero.\\
	\hline BITSTREAM & 15:8 & ro & 0x0 & Bitstream version\\
	\hline BOARD & 7:0 & ro & 0x0 & Axiom board model\\
\end{regdetails}


\subsubsection{IFNUMBER} \label{reg:ifnumber}
\begin{regdescription}
	Name			\> IFNUMBER\\
	Relative Address	\> 0xA0140000\\
	Width			\> 32 bits\\
	Access Type		\> ro\\
	Init Value		\> 0x00\\
	Description		\> Number of interface\\
\end{regdescription}
\begin{regdetails}
	\hline reserved & 31:3 & ro & 0x0 & Reserved. Writes are ignored, read
	data is zero.\\
	\hline IFNUMBER & 2:0 & ro & 0x2 & Number of interface which are present
	on a node.\\
\end{regdetails}


\subsubsection{IFINFO\_BASE\_1} \label{reg:ifinfo_base_1}
\begin{regdescription}
	Name			\> IFINFO\_BASE\_1\\
	Relative Address	\> 0xA0150000\\
	Width (single row)	\> 8 bits\\
	Access Type		\> ro\\
	Init Value		\> 0x00\\
	Description		\> Interface 1 (Rx\_1) status table\\
\end{regdescription}
\begin{regdetails}
	\hline reserved & 7:3 & ro & 0x0 & Reserved. Writes are ignored, read
	data is zero.\\
	\hline CONNECTED & 2 & ro & 0x0 & Interface connected status.\\
               & & & & 1b = interface is physically connected to another board.\\
               & & & & 0b = interface is disconnected.\\
	\hline RX & 1 & ro & 0x0 & Interface RX functionality.\\
               & & & & This bit is set when the interface can work as RX.\\
	\hline TX & 0 & ro & 0x0 & Interface TX functionality.\\
               & & & & This bit is set when the interface can work as TX.\\
\end{regdetails}
\subsubsection{IFINFO\_BASE\_2} \label{reg:ifinfo_base_2}
\begin{regdescription}
	Name			\> IFINFO\_BASE\_2\\
	Relative Address	\> 0xA0160000\\
	Width (single row)	\> 8 bits\\
	Access Type		\> ro\\
	Init Value		\> 0x00\\
	Description		\> Interface 2 (Tx\_1) status table\\
\end{regdescription}
\begin{regdetails}
	\hline reserved & 7:3 & ro & 0x0 & Reserved. Writes are ignored, read
	data is zero.\\
	\hline CONNECTED & 2 & ro & 0x0 & Interface connected status.\\
               & & & & 1b = interface is physically connected to another board.\\
               & & & & 0b = interface is disconnected.\\
	\hline RX & 1 & ro & 0x0 & Interface RX functionality.\\
               & & & & This bit is set when the interface can work as RX.\\
	\hline TX & 0 & ro & 0x0 & Interface TX functionality.\\
               & & & & This bit is set when the interface can work as TX.\\
\end{regdetails}


\section{Control Registers} \label{mod:control}
\begin{regdescription}
	Module Name 	\> Control Registers\\
	Description 	\> Registers used to control the device\\
\end{regdescription}

\subsection{Register summary}
\begin{regsummary}
	\hline \nameref{reg:nodeid} & 0xA00A0000 & 32 & rw & 0x0 & Node ID
	register \\
	\hline \nameref{reg:dma_start} & 0xA0080000 & 64 & wo & 0x0 & DMA start
	register \\
	\hline \nameref{reg:dma_end} & 0x00000050 & 64 & wo & 0x0 & DMA end
	register \\
\end{regsummary}


\subsubsection{NODEID} \label{reg:nodeid}
\begin{regdescription}
	Name			\> NODEID\\
	Relative Address	\> 0xA00A0000\\
	Width			\> 32 bits\\
	Access Type		\> rw\\
	Init Value		\> 0x00000000\\
	Description		\> Node ID register\\
\end{regdescription}
\begin{regdetails}
	\hline reserved & 31:8 & rw & 0x0 & Reserved. Writes are ignored, read
	data is zero.\\
	\hline NODEID & 7:0 & rw & 0x0 & Defines the ID of a local node.\\
\end{regdetails}


\subsubsection{DMA\_START} \label{reg:dma_start}
\begin{regdescription}
	Name			\> DMA\_START\\
	Relative Address	\> 0xA0080000 (+0x8 for second channel)\\
	Width			\> 2 Channels * 32 bits\\
	Access Type		\> wo\\
	Init Value		\> 0x00000000\\
	Description		\> DMA start register\\
\end{regdescription}
\begin{regdetails}
	\hline DMA\_START\_LOW & 31:0 & wo & 0x0 & Lower Half of the Physical memory address of the
	first byte of DMA zone that can be accessed from the NIC. This zone
	must contain the LONG TX and RX buffers and RDMA zone\\
	\hline DMA\_START\_HIGH & 63:32 & wo & 0x0 & Upper Half of the Physical memory address of the
	first byte of DMA zone that can be accessed from the NIC.\\
\end{regdetails}


\subsubsection{DMA\_END} \label{reg:dma_end}
\begin{regdescription}
	Name			\> DMA\_END\\
	Relative Address	\> 0xA0070000 (+0x8 for second channel)\\
	Width			\> 2 Channels * 32 bits\\
	Access Type		\> wo\\
	Init Value		\> 0x00000000\\
	Description		\> DMA end register\\
\end{regdescription}
\begin{regdetails}
	\hline DMA\_END\_LOW & 31:0 & wo & 0x0 & Lower Half of the Physical memory address of the
	last byte of DMA zone that can be accessed from the NIC. This zone
	must contain the LONG TX and RX buffers and RDMA zone\\
	\hline DMA\_END\_HIGH & 63:32 & wo & 0x0 & Upper Half of the Physical memory address of the
	last byte of DMA zone that can be accessed from the NIC.\\
\end{regdetails}

\section{Interrupt Registers} \label{mod:interrupt}
\begin{regdescription}
	Module Name 	\> Interrupt Registers\\
	Description 	\> Registers used to handle interrupts.\\
	                \> If the interrupts are enabled, they are generated in these cases:\\
	                \> - RX queues. Interrupt generated when the queue is empty and new\\
	                \>      descriptor is enqueued by the HW. [transition empty,
	                not empty]\\
	                \> - TX queues. Interrupt generated when the queue is full and a descriptor\\
	                \>      is dequeued by the HW. [transition full, not full]\\
\end{regdescription}

\subsection{Register summary}
\begin{regsummary}
	\hline \nameref{reg:mskirq} & 0xA0090000 & 32 & rw & 0x0 & Mask
	Interrupt Register\\
	\hline \nameref{reg:pndirq} & 0xA00B0000 & 32 & rw & 0x0 & Pending
	Interrupt Register\\
\end{regsummary}

\subsubsection{MSKIRQ} \label{reg:mskirq}
\begin{regdescription}
	Name			\> MSKIRQ\\
	Relative Address	\> 0xA0090000\\
	Width			\> 32 bits\\
	Access Type		\> rw\\
	Init Value		\> 0x00000000\\
	Description		\> Mask Interrupt Register. Used to
	                           enable/disable interrupts\\
\end{regdescription}
\begin{regdetails}
	\hline reserved & 31:4 & rw & 0x0 & Reserved. Writes are ignored, read
	data is zero.\\
	\hline IRQ\_RDMA\_RX & 3 & rw & 0x0 & Interrupt RDMA RX queue.\\
               & & & & 1b = RDMA RX interrupt is enabled.\\
               & & & & 0b = RDMA RX interrupt is disabled.\\
	\hline IRQ\_RDMA\_TX & 2 & rw & 0x0 & Interrupt RDMA TX queue.\\
               & & & & 1b = RDMA TX interrupt is enabled.\\
               & & & & 0b = RDMA TX interrupt is disabled.\\
	\hline IRQ\_RAW\_RX & 1 & rw & 0x0 & Interrupt RAW RX queue.\\
               & & & & 1b = RAW RX interrupt is enabled.\\
               & & & & 0b = RAW RX interrupt is disabled.\\
	\hline IRQ\_RAW\_TX & 0 & rw & 0x0 & Interrupt RAW TX queue.\\
               & & & & 1b = RAW TX interrupt is enabled.\\
               & & & & 0b = RAW TX interrupt is disabled.\\
\end{regdetails}

\subsubsection{PNDIRQ} \label{reg:pndirq}
\begin{regdescription}
	Name			\> PNDIRQ\\
	Relative Address	\> 0xA00B0000\\
	Width			\> 32 bits\\
	Access Type		\> rw\\
	Init Value		\> 0x00000000\\
	Description		\> Pending Interrupt Register. Used to report
	                           the cause of the interrupt \\
	                        \> and send back an ACK, when the SW handles it.\\
\end{regdescription}
\begin{regdetails}
	\hline reserved & 31:4 & rw & 0x0 & Reserved. Read data is zero.\\
	\hline IRQ\_RDMA\_RX & 3 & rw & 0x0 & Interrupt RDMA RX queue.\\
               & & & & This bit is set by the HW to 1b when the RDMA RX queue
               generates an interrupt.\\
               & & & & This bit is reset by the SW to 0b when the RDMA RX queue
               interrupt is acknowledged.\\
	\hline IRQ\_RDMA\_TX & 2 & rw & 0x0 & Interrupt RDMA TX queue.\\
               & & & & This bit is set by the HW to 1b when the RDMA TX queue
               generates an interrupt.\\
               & & & & This bit is reset by the SW to 0b when the RDMA TX queue
               interrupt is acknowledged.\\
	\hline IRQ\_RAW\_RX & 1 & rw & 0x0 & Interrupt RAW RX queue.\\
               & & & & This bit is set by the HW to 1b when the RAW RX queue
               generates an interrupt.\\
               & & & & This bit is reset by the SW to 0b when the RAW RX queue
               interrupt is acknowledged.\\
	\hline IRQ\_RAW\_TX & 0 & rw & 0x0 & Interrupt RAW TX queue.\\
               & & & & This bit is set by the HW to 1b when the RAW TX queue
               generates an interrupt.\\
               & & & & This bit is reset by the SW to 0b when the RAW TX queue
               interrupt is acknowledged.\\
\end{regdetails}


\section{Routing Registers} \label{mod:routing}
\begin{regdescription}
	Module Name 	\> Routing Registers\\
	Description 	\> Registers used to manage the routing table\\
    \> All four Routing Tables should have the same values\\
\end{regdescription}

\subsection{Register summary}
\begin{regsummary}
    \hline \nameref{reg:PHY1_routing_base} & 0xA0010000+n & 8 & rw & 0x00 & Routing
    table base register\\
                            & n=0...255 & & & & \\
                            \hline \nameref{reg:rx_routing_base} & 0xA0014000+n & 8 & rw & 0x00 & Routing
    table base register\\
                            & n=0...255 & & & & \\
                            \hline \nameref{reg:tx_dma_routing_base} & 0xA0016000+n & 8 & rw & 0x00 & Routing
    table base register\\
                            & n=0...255 & & & & \\
                            \hline \nameref{reg:tx_raw_routing_base} & 0xA0018000+n & 8 & rw & 0x00 & Routing
    table base register\\
                            & n=0...255 & & & & \\
\end{regsummary}

\subsubsection{PHY1\_ROUTING\_BASE} \label{reg:PHY1_routing_base}
\begin{regdescription}
	Name			\> PHY1\_ROUTING\_BASE\\
	Relative Address	\> 0xA0010000+n, n=0...255\\
	Width (single row)	\> 8 bits\\
	Access Type		\> rw\\
	Init Value		\> 0x00000000\\
	Description		\> Routing table base register\\
	                        \> This array of registers represents the
	                        routing table.\\
	                        \> The array index represents the destination
	                        node id.\\
\end{regdescription}
\begin{regdetails}
	\hline ENABLED\_IF & 7:0 & rw & 0x0 & ID of the interface to reach the
	node n (array index). Value 0x0 is the NI and value 0x1 is the Tx\_1 IF. If the node is not reachable, the SW must write
	0xFF in this register.\\
\end{regdetails}

\subsubsection{RX\_ROUTING\_BASE} \label{reg:rx_routing_base}
\begin{regdescription}
	Name			\> RX\_ROUTING\_BASE\\
	Relative Address	\> 0xA0014000+n, n=0...255\\
	Width (single row)	\> 8 bits\\
	Access Type		\> rw\\
	Init Value		\> 0x00000000\\
	Description		\> Routing table base register\\
	                        \> This array of registers represents the
	                        routing table.\\
	                        \> The array index represents the destination
	                        node id.\\
\end{regdescription}
\begin{regdetails}
	\hline ENABLED\_IF & 7:0 & rw & 0x0 & ID of the interface to reach the
	node n (array index). Value 0x0 is the NI and value 0x1 is the Tx\_1 IF. If the node is not reachable, the SW must write
	0xFF in this register.\\
\end{regdetails}

\subsubsection{TX\_DMA\_ROUTING\_BASE} \label{reg:tx_dma_routing_base}
\begin{regdescription}
	Name			\> TX\_DMA\_ROUTING\_BASE\\
	Relative Address	\> 0xA0016000+n, n=0...255\\
	Width (single row)	\> 8 bits\\
	Access Type		\> rw\\
	Init Value		\> 0x00000000\\
	Description		\> Routing table base register\\
	                        \> This array of registers represents the
	                        routing table.\\
	                        \> The array index represents the destination
	                        node id.\\
\end{regdescription}
\begin{regdetails}
	\hline ENABLED\_IF & 7:0 & rw & 0x0 & ID of the interface to reach the
	node n (array index). Value 0x0 is the NI and value 0x1 is the Tx\_1 IF. If the node is not reachable, the SW must write
	0xFF in this register.\\
\end{regdetails}

\subsubsection{TX\_RAW\_ROUTING\_BASE} \label{reg:tx_raw_routing_base}
\begin{regdescription}
	Name			\> TX\_RAW\_ROUTING\_BASE\\
	Relative Address	\> 0xA0018000+n, n=0...255\\
	Width (single row)	\> 8 bits\\
	Access Type		\> rw\\
	Init Value		\> 0x00000000\\
	Description		\> Routing table base register\\
	                        \> This array of registers represents the
	                        routing table.\\
	                        \> The array index represents the destination
	                        node id.\\
\end{regdescription}
\begin{regdetails}
	\hline ENABLED\_IF & 7:0 & rw & 0x0 & ID of the interface to reach the
	node n (array index). Value 0x0 is the NI and value 0x1 is the Tx\_1 IF. If the node is not reachable, the SW must write
	0xFF in this register.\\
\end{regdetails}

\subsection{Queues Registers} \label{mod:queue}
\begin{regdescription}
	Module Name 	\> Queues Registers\\
	Description 	\> Registers used to manage the descriptor queues\\
\end{regdescription}

\subsection{Register summary}
\begin{regsummary}
	\hline \nameref{reg:raw_tx_status} & 0xA0120000+0x0c & 32 & ro & 0x0 & RAW TX
	queue status register\\
        \hline \nameref{reg:raw_tx_desc} & 0xA0130000 & 32 & wo & 0x0 & RAW TX
        queue descriptor register\\
	\hline \nameref{reg:raw_rx_status} & 0xA00E0000+0x1c & 32 & ro & 0x0 & RAW RX
	queue status register\\
        \hline \nameref{reg:raw_rx_desc} & 0xA00F0000+0x1000 & 32 & ro & 0x0 & RAW RX
        queue descriptor register\\
	\hline \nameref{reg:rdma_tx_status} & 0xA0100000+0x0c & 32 & ro & 0x0 & RDMA
	TX queue status register\\
        \hline \nameref{reg:rdma_tx_desc} & 0xA0110000 & 32 & wo & 0x0 & RDMA TX
        queue descriptor register\\
	\hline \nameref{reg:rdma_rx_status} & 0xA00C0000+0x1c & 32 & ro & 0x0 & RDMA
	RX queue status register\\
        \hline \nameref{reg:rdma_rx_desc} & 0xA00D0000+0x1000 & 32 & ro & 0x0 & RDMA RX
        queue descriptor register\\
        \hline \nameref{reg:long_buf_base} & 0xA0012000+8*n & 64 & rw & 0x0 & LONG
        buffers base register\\
            & n=0...31 & & & & \\
\end{regsummary}

\subsubsection{RAW\_TX\_STATUS} \label{reg:raw_tx_status}
\begin{regdescription}
	Name			\> RAW\_TX\_STATUS\\
	Relative Address	\> 0xA0120000+0x0c\\
	Width			\> 32 bits\\
	Access Type		\> ro\\
	Init Value		\> 0x00000000\\
	Description		\> RAW TX queue status register\\
\end{regdescription}
\begin{regdetails}
	\hline QSTATUS\_AVAIL & 31:0 & ro & 0x0 & The Vacancy of the Tx FIFO. When greater than 0, in the RAW TX
	queue there is space available for new descriptors from the SW.\\
\end{regdetails}

\subsubsection{RAW\_TX\_DESC} \label{reg:raw_tx_desc}
\begin{regdescription}
	Name			\> RAW\_TX\_DESC\\
	Relative Address	\> 0xA0130000\\
	Width 	                \> 32 bits\\
	Access Type		\> wo\\
	Init Value		\> 0x00000000\\
	Description		\> RAW TX queue descriptor register\\
	                        \> These registers represent the descriptor of
	                        the RAW TX queue.\\
	                        \> The queue is handled internally by the
	                        hardware.\\
	                        \> The software can push new elements, writing
	                        these registers\\
\end{regdescription}
\begin{regdetails}
	\hline PAYLOAD & 2079:40 & ro & 0x0 & Payload of this message\\
	\hline reserved & 39:32 & wo & 0x0 & Reserved. Writes are ignored, read
	data is zero.\\
	\hline PAYLOAD\_SIZE & 31:24 & wo & 0x0 & Payload size of this message\\
	\hline MSG\_ID & 23:16 & wo & 0x0 & Unique ID for this message spcified
	by the software\\
	\hline DESTINATION & 15:8 & wo & 0x0 & Receiver node id or local
	interface to send message to the neighbour \\
	\hline S & 7 & wo & 0x0 & S bit. Ignored in RAW message\\
	\hline ERROR & 6 & wo & 0x0 & Error bit. Ignored in TX message\\
	\hline PORT & 5:3 & wo & 0x0 & Message port.\\
        \hline TYPE & 2:0 & wo & 0x0 & Message types.\\
                    & & & & (ACK = 0, INIT = 1, RAW\_NEIGHBOUR = 2, RAW\_DATA = 3, LONG\_DATA = 4, RDMA\_WRITE = 5, RDMA\_READ = 6, RDMA\_RESPONSE = 7)\\
\end{regdetails}

\subsubsection{RAW\_RX\_STATUS} \label{reg:raw_rx_status}
\begin{regdescription}
	Name			\> RAW\_RX\_STATUS\\
	Relative Address	\> 0xA00E0000+0x1c\\
	Width			\> 32 bits\\
	Access Type		\> ro\\
	Init Value		\> 0x00000000\\
	Description		\> RAW RX queue status register\\
\end{regdescription}
\begin{regdetails}
	\hline QSTATUS\_AVAIL & 31:0 & ro & 0x0 & The Occupancy of the Rx FIFO. When greater than 0, in the RAW RX
	queue there are descriptors available to be read from the SW.\\
\end{regdetails}

\subsubsection{RAW\_RX\_DESC} \label{reg:raw_rx_desc}
\begin{regdescription}
	Name			\> RAW\_RX\_DESC\\
	Relative Address	\> 0xA00F0000+0x1000\\
	Width (single row)	\> 32 bits\\
	Access Type		\> ro\\
	Init Value		\> 0x00000000\\
	Description		\> RAW RX queue descriptor register\\
	                        \> These registers represent the descriptor of
	                        the RAW RX queue.\\
	                        \> The queue is handled internally by the
	                        hardware.\\
	                        \> The software can pop new elements, reading
	                        these registers\\
\end{regdescription}
\begin{regdetails}
	\hline PAYLOAD & 2079:40 & ro & 0x0 & Payload of this message\\
	\hline reserved & 39:32 & ro & 0x0 & Reserved. Writes are ignored, read
	data is zero.\\
	\hline PAYLOAD\_SIZE & 31:24 & ro & 0x0 & Payload size of this message\\
	\hline MSG\_ID & 23:16 & ro & 0x0 & Unique ID for this message spcified
	by the software\\
	\hline SOURCE & 15:8 & ro & 0x0 & Sender node id or local interface
	where neighbour message is received\\
	\hline S & 7 & ro & 0x0 & S bit. Ignored in RAW message\\
	\hline ERROR & 6 & ro & 0x0 & Error bit. When set to 1b, there is an error.\\
	\hline PORT & 5:3 & ro & 0x0 & Message port.\\
        \hline TYPE & 2:0 & ro & 0x0 & Message types.\\
                    & & & & (ACK = 0, INIT = 1, RAW\_NEIGHBOUR = 2, RAW\_DATA = 3, LONG\_DATA = 4, RDMA\_WRITE = 5, RDMA\_READ = 6, RDMA\_RESPONSE = 7)\\
\end{regdetails}

\subsubsection{RDMA\_TX\_STATUS} \label{reg:rdma_tx_status}
\begin{regdescription}
	Name			\> RDMA\_TX\_STATUS\\
	Relative Address	\> 0xA0100000+0x0c\\
	Width			\> 32 bits\\
	Access Type		\> ro\\
	Init Value		\> 0x00000000\\
	Description		\> RDMA queue status register\\
\end{regdescription}
\begin{regdetails}
	\hline QSTATUS\_AVAIL & 31:0 & ro & 0x0 & The Vacancy of the Tx FIFO. When greater than 0, in the DMA TX
	queue there is space available for new descriptors from the SW.\\
\end{regdetails}

\subsubsection{RDMA\_TX\_DESC} \label{reg:rdma_tx_desc}
\begin{regdescription}
	Name			\> RDMA\_TX\_DESC\\
	Relative Address	\> 0xA0110000\\
	Width 	                \> 32 bits\\
	Access Type		\> wo\\
	Init Value		\> 0x00000000\\
	Description		\> RDMA TX queue descriptor register\\
	                        \> These registers represent the descriptor of
	                        the RDMA TX queue.\\
	                        \> The queue is handled internally by the
	                        hardware.\\
	                        \> The software can push new elements, writing
	                        these registers\\
\end{regdescription}
\begin{regdetails}
        \hline DST\_ADDR & 103:72 & wo & 0x0 & Destination memory address, not used for LONG\_DATA \\
        \hline SRC\_ADDR & 71:40 & wo & 0x0 & Source memory address\\
	\hline PAYLOAD\_SIZE & 39:24 & wo & 0x0 & Payload size of this RDMA\\
	\hline MSG\_ID & 23:16 & wo & 0x0 & Unique ID for this message spcified
	by the software\\
	\hline DESTINATION & 15:8 & wo & 0x0 & Receiver node id\\
	\hline S & 7 & wo & 0x0 & S bit. Ignored in TX message\\
	\hline ERROR & 6 & wo & 0x0 & Error bit. Ignored in TX message\\
	\hline PORT & 5:3 & wo & 0x0 & Message port.\\
        \hline TYPE & 2:0 & wo & 0x0 & Message types.\\
                    & & & & (ACK = 0, INIT = 1, RAW\_NEIGHBOUR = 2, RAW\_DATA = 3, LONG\_DATA = 4, RDMA\_WRITE = 5, RDMA\_READ = 6, RDMA\_RESPONSE = 7)\\
\end{regdetails}

\subsubsection{RDMA\_RX\_STATUS} \label{reg:rdma_rx_status}
\begin{regdescription}
	Name			\> RDMA\_RX\_STATUS\\
	Relative Address	\> 0xA00C0000+0x1c\\
	Width			\> 32 bits\\
	Access Type		\> ro\\
	Init Value		\> 0x00000000\\
	Description		\> RDMA RX queue status register\\
\end{regdescription}
\begin{regdetails}
	\hline QSTATUS\_AVAIL & 31:0 & ro & 0x0 & The Occupancy of the Rx FIFO. When greater than 0, in the DMA RX
	queue there are descriptors available to be read from the SW.\\
\end{regdetails}

\subsubsection{RDMA\_RX\_DESC} \label{reg:rdma_rx_desc}
\begin{regdescription}
	Name			\> RDMA\_RX\_DESC\\
	Relative Address	\> 0xA00D0000+0x1000\\
	Width                   \> 32 bits\\
	Access Type		\> ro\\
	Init Value		\> 0x00000000\\
	Description		\> RDMA RX queue descriptor register\\
	                        \> These registers represent the descriptor of
	                        the RAW RX queue.\\
	                        \> The queue is handled internally by the
	                        hardware.\\
	                        \> The software can pop new elements, reading
	                        these registers\\
\end{regdescription}
\begin{regdetails}
        \hline SRC\_ADDR & 71:40 & ro & 0x0 & Destination memory address, used only when S = 0b\\
	\hline PAYLOAD\_SIZE & 39:24 & ro & 0x0 & Payload size of this RDMA\\
	\hline MSG\_ID & 23:16 & ro & 0x0 & Unique ID for this message spcified
	by the software\\
	\hline SOURCE & 15:8 & ro & 0x0 & Sender node id\\
	\hline S & 7 & ro & 0x0 & S bit. When set to 1b, this message is an ACK
	of message previously sent by this node\\
	\hline ERROR & 6 & ro & 0x0 & Error bit. When set to 1b, the ACK
	contains an error from the receive.\\
	\hline PORT & 5:3 & ro & 0x0 & Message port.\\
        \hline TYPE & 2:0 & ro & 0x0 & Message types.\\
                    & & & & (ACK = 0, INIT = 1, RAW\_NEIGHBOUR = 2, RAW\_DATA = 3, LONG\_DATA = 4, RDMA\_WRITE = 5, RDMA\_READ = 6, RDMA\_RESPONSE = 7)\\
\end{regdetails}

\subsubsection{LONG\_BUF\_BASE[n]} \label{reg:long_buf_base}
\begin{regdescription}
	Name			\> LONG\_BUF\_BASE[n]\\
	Relative Address	\> 0xA0012000+8*n, n=0...31\\
	Width (single row)	\> 64 bits\\
	Access Type		\> rw\\
	Init Value		\> 0x00000000\\
	Description		\> LONG buffers base register\\
	                        \> This array of registers represents the
	                        address of buffers available\\
	                        \> for receiving the LONG messages.\\
\end{regdescription}
\begin{regdetails}
	\hline FLAGS & 63:56 & rw & 0x0 & When the software read this buffer, it
	must write 1b to signal to the HW that this buffer is now free.\\
	\hline MSG\_ID & 55:48 & rw & 0x0 & The HW must put the MSG\_ID of
	message that uses this buffer.\\
	\hline SIZE & 47:32 & rw & 0x0 & Size of buffer \\
	\hline ADDRESS & 31:0 & rw & 0x0 & Address of buffer \\
\end{regdetails}
\end{document}
