\documentclass[10pt,a4paper]{paper}
\usepackage[latin1]{inputenc}
\usepackage[margin=2cm]{geometry}
\usepackage{amsmath}
\usepackage{amsfonts}
\usepackage{amssymb}
\usepackage{graphicx}
\usepackage{tabularx}
\usepackage{environ}
\usepackage[colorlinks,linkcolor=blue]{hyperref}

\NewEnviron{regdescription}
	{\begin{tabbing}
	\hspace{150pt}\=\kill
		\BODY
	\end{tabbing}}

\NewEnviron{regglobalsummary}
	{\begin{tabularx}{\textwidth}{|l|l|l|l|X|}
	        \hline \textbf{Register Name} & \textbf{Address} &
	        \textbf{Width} & \textbf{Type} & \textbf{Description} \\
		\BODY
	\hline
	\end{tabularx}}

\NewEnviron{regsummary}
	{\begin{tabularx}{\textwidth}{|l|l|l|l|l|X|}
	        \hline \textbf{Register Name} & \textbf{Address} &
	        \textbf{Width} & \textbf{Type} & \textbf{Init Val} &
	        \textbf{Description} \\
		\BODY
	\hline
	\end{tabularx}}

\NewEnviron{regdetails}
	{\begin{tabularx}{\textwidth}{|l|l|l|l|X|}
	        \hline \textbf{Field Name} & \textbf{Bit(s)} & \textbf{Type} &
	        \textbf{Init Val} & \textbf{Description} \\
	 	\BODY
	\hline 
	\end{tabularx}}

\newcommand{\versionapi}{v0.6-RC1}

% Title Page
\title{Axiom NIC Datasheet}
\subtitle{\versionapi - \today}
\author{Evidence SRL}

\begin{document}
\maketitle


%\chapter{Axiom NIC}
\section{AXIOM NIC Registers}
\begin{regdescription}
	Total size      \> 1568 bytes\\
\end{regdescription}

\subsection{Register summary}
\begin{regglobalsummary}
        \hline \textbf{\nameref{mod:status}} & & & & \\
                                             & & & & \\
        \hline \nameref{reg:version} & 0x00000000 & 32 & ro & Version register\\
        \hline \nameref{reg:status} & 0x00000004 & 32 & ro & Status register\\
	\hline \nameref{reg:ifnumber} & 0x00000008 & 32 & ro & Number of
	interface\\
        \hline \nameref{reg:ifinfo_base} & 0x00000010+n, & 8 & ro & Interface
        status table\\ & n=0...3 & & & \\
        \hline \textbf{\nameref{mod:control}} & & & & \\
                                              & & & & \\
	\hline \nameref{reg:control} & 0x00000040 & 32 & rw & Control register\\
	\hline \nameref{reg:nodeid} & 0x00000044 & 32 & rw & Node ID register \\
        \hline \textbf{\nameref{mod:interrupt}} & & & & \\
                                                & & & & \\
	\hline \nameref{reg:ackirq} & 0x00000060 & 32 & wo & ACK Interrupt
	Register\\
	\hline \nameref{reg:mskirq} & 0x00000064 & 32 & rw & Mask Interrupt
	Register\\
	\hline \nameref{reg:pndirq} & 0x00000068 & 32 & ro & Pending Interrupt
	Register\\
        \hline \textbf{\nameref{mod:routing}} & & & & \\
                                              & & & & \\
        \hline \nameref{reg:routing_base} & 0x00000100+n & 8 & rw & Routing
        table base register\\ & n=0...255 & & & \\
        \hline \textbf{\nameref{mod:queue}} & & & & \\
                                            & & & & \\
	\hline \nameref{reg:raw_tx_status} & 0x00000300 & 32 & ro & RAW TX queue
	status register\\
        \hline \nameref{reg:raw_tx_desc} & 0x00000400 & 1056 & wo & RAW TX queue
        descriptor register\\
	\hline \nameref{reg:raw_rx_status} & 0x00000310 & 32 & ro & RAW RX queue
	status register\\
        \hline \nameref{reg:raw_rx_desc} & 0x00000500 & 1056 & ro & RAW RX queue
        descriptor register\\
	\hline \nameref{reg:rdma_tx_status} & 0x00000320 & 32 & ro & RDMA TX
	queue status register\\
        \hline \nameref{reg:rdma_tx_desc} & 0x00000600 & 96 & wo & RDMA TX queue
        descriptor register\\
	\hline \nameref{reg:rdma_rx_status} & 0x00000330 & 32 & ro & RDMA RX
	queue status register\\
        \hline \nameref{reg:rdma_rx_desc} & 0x00000610 & 96 & ro & RDMA RX queue
        descriptor register\\
\end{regglobalsummary}


\section{Status Registers} \label{mod:status}
\begin{regdescription}
	Module Name 	\> Status Registers\\
	Description 	\> Registers used to check the status of the device\\
\end{regdescription}

\subsection{Register summary}
\begin{regsummary}
	\hline \nameref{reg:version} & 0x00000000 & 32 & ro & 0x0 & Version
	register\\
	\hline \nameref{reg:status} & 0x00000004 & 32 & ro & 0x0 & Status
	register\\
	\hline \nameref{reg:ifnumber} & 0x00000008 & 32 & ro & 0x0 & Number of
	interface\\
        \hline \nameref{reg:ifinfo_base} & 0x00000010+n, & 8 & ro & 0x0 &
        Interface status table\\
                               & n=0...7 & & & & \\
\end{regsummary}


\subsubsection{VERSION} \label{reg:version}
\begin{regdescription}
	Name			\> VERSION\\
	Relative Address	\> 0x00000000\\
	Width			\> 32 bits\\
	Access Type		\> ro\\
	Init Value		\> 0x00000000\\
	Description		\> Version register\\
\end{regdescription}
\begin{regdetails}
	\hline reserved & 31:16 & ro & 0x0 & Reserved. Writes are ignored, read
	data is zero.\\
	\hline BITSTREAM & 15:8 & ro & 0x0 & Bitstream version\\
	\hline BOARD & 7:0 & ro & 0x0 & Axiom board model\\
\end{regdetails}


\subsubsection{STATUS} \label{reg:status}
\begin{regdescription}
	Name			\> STATUS\\
	Relative Address	\> 0x00000004\\
	Width			\> 32 bits\\
	Access Type		\> ro\\
	Init Value		\> 0x00000000\\
	Description		\> Status register\\
\end{regdescription}
\begin{regdetails}
	\hline reserved & 31:1 & ro & 0x0 & Reserved. Writes are ignored, read
	data is zero.\\
	\hline TBD & 0 & rw & 0x0 & TBD\\
\end{regdetails}


\subsubsection{IFNUMBER} \label{reg:ifnumber}
\begin{regdescription}
	Name			\> IFNUMBER\\
	Relative Address	\> 0x00000008\\
	Width			\> 32 bits\\
	Access Type		\> ro\\
	Init Value		\> 0x00\\
	Description		\> Number of interface\\
\end{regdescription}
\begin{regdetails}
	\hline reserved & 31:3 & ro & 0x0 & Reserved. Writes are ignored, read
	data is zero.\\
	\hline IFNUMBER & 2:0 & ro & 0x2 & Number of interface which are present
	on a node.\\
\end{regdetails}


\subsubsection{IFINFO\_BASE[n]} \label{reg:ifinfo_base}
\begin{regdescription}
	Name			\> IFINFO\_BASE[n]\\
	Relative Address	\> 0x00000010+n, n=0...7\\
	Width (single row)	\> 8 bits\\
	Access Type		\> ro\\
	Init Value		\> 0x00\\
	Description		\> Interface status table\\
\end{regdescription}
\begin{regdetails}
	\hline reserved & 7:3 & ro & 0x0 & Reserved. Writes are ignored, read
	data is zero.\\
	\hline CONNECTED & 2 & ro & 0x0 & Interface connected status.\\
               & & & & 1b = interface is physically connected to another board.\\
               & & & & 0b = interface is disconnected.\\
	\hline RX & 1 & ro & 0x0 & Interface RX functionality.\\
               & & & & This bit is set when the interface can work as RX.\\
	\hline TX & 0 & ro & 0x0 & Interface TX functionality.\\
               & & & & This bit is set when the interface can work as TX.\\
\end{regdetails}



\section{Control Registers} \label{mod:control}
\begin{regdescription}
	Module Name 	\> Control Registers\\
	Description 	\> Registers used to control the device\\
\end{regdescription}

\subsection{Register summary}
\begin{regsummary}
	\hline \nameref{reg:control} & 0x00000040 & 32 & rw & 0x0 & Control
	register \\
	\hline \nameref{reg:nodeid} & 0x00000044 & 32 & rw & 0x0 & Node ID
	register \\
\end{regsummary}


\subsubsection{CONTROL} \label{reg:control}
\begin{regdescription}
	Name			\> CONTROL\\
	Relative Address	\> 0x00000040\\
	Width			\> 32 bits\\
	Access Type		\> rw\\
	Init Value		\> 0x00000000\\
	Description		\> Control register. Used to enable local
	                           transmission ACKs\\
	                        \> and/or the PHY loopback mode configuration.\\
\end{regdescription}
\begin{regdetails}
	\hline reserved & 31:1 & rw & 0x0 & Reserved. Writes are ignored, read
	data is zero.\\
	\hline LOOPBACK & 0 & rw & 0x0 & When set to 1b, all the interfaces are
	in loopback mode.\\
\end{regdetails}


\subsubsection{NODEID} \label{reg:nodeid}
\begin{regdescription}
	Name			\> NODEID\\
	Relative Address	\> 0x00000044\\
	Width			\> 32 bits\\
	Access Type		\> rw\\
	Init Value		\> 0x00000000\\
	Description		\> Node ID register\\
\end{regdescription}
\begin{regdetails}
	\hline NODEID & 7:0 & rw & 0x0 & Defines the ID of a local node.\\
\end{regdetails}



\section{Interrupt Registers} \label{mod:interrupt}
\begin{regdescription}
	Module Name 	\> Interrupt Registers\\
	Description 	\> Registers used to handle interrupts\\
\end{regdescription}

\subsection{Register summary}
\begin{regsummary}
	\hline \nameref{reg:ackirq} & 0x00000060 & 32 & wo & 0x0 & ACK Interrupt
	Register\\
	\hline \nameref{reg:mskirq} & 0x00000064 & 32 & rw & 0x0 & Mask
	Interrupt Register\\
	\hline \nameref{reg:pndirq} & 0x00000068 & 32 & ro & 0x0 & Pending
	Interrupt Register\\
\end{regsummary}

\subsubsection{ACKIRQ} \label{reg:ackirq}
\begin{regdescription}
	Name			\> ACKIRQ\\
	Relative Address	\> 0x00000060\\
	Width			\> 32 bits\\
	Access Type		\> wo\\
	Init Value		\> 0x00000000\\
	Description		\> ACK Interrupt Register. Used to send an ack
	                           for the interrupts\\
\end{regdescription}
\begin{regdetails}
	\hline reserved & 31:4 & wo & 0x0 & Reserved. Writes are ignored.\\
	\hline IRQ\_RDMA\_RX & 3 & wo & 0x0 & Interrupt RDMA RX queue.\\
               & & & & When set to 1b, the RDMA TX interrupt is acknowledged.\\
	\hline IRQ\_RDMA\_TX & 2 & wo & 0x0 & Interrupt RDMA TX queue.\\
               & & & & When set to 1b, the RDMA RX interrupt is acknowledged.\\
	\hline IRQ\_RAW\_RX & 1 & wo & 0x0 & Interrupt RAW RX queue.\\
               & & & & When set to 1b, the RAW TX interrupt is acknowledged.\\
	\hline IRQ\_RAW\_TX & 0 & wo & 0x0 & Interrupt RAW TX queue.\\
               & & & & When set to 1b, the RAW RX interrupt is acknowledged.\\
\end{regdetails}

\subsubsection{MSKIRQ} \label{reg:mskirq}
\begin{regdescription}
	Name			\> MSKIRQ\\
	Relative Address	\> 0x00000064\\
	Width			\> 32 bits\\
	Access Type		\> rw\\
	Init Value		\> 0x00000000\\
	Description		\> Mask Interrupt Register. Used to
	                           enable/disable interrupts\\
\end{regdescription}
\begin{regdetails}
	\hline reserved & 31:4 & rw & 0x0 & Reserved. Writes are ignored, read
	data is zero.\\
	\hline IRQ\_RDMA\_RX & 3 & rw & 0x0 & Interrupt RDMA RX queue.\\
               & & & & 1b = RDMA RX interrupt is enabled.\\
               & & & & 0b = RDMA RX interrupt is disabled.\\
	\hline IRQ\_RDMA\_TX & 2 & rw & 0x0 & Interrupt RDMA TX queue.\\
               & & & & 1b = RDMA TX interrupt is enabled.\\
               & & & & 0b = RDMA TX interrupt is disabled.\\
	\hline IRQ\_RAW\_RX & 1 & rw & 0x0 & Interrupt RAW RX queue.\\
               & & & & 1b = RAW RX interrupt is enabled.\\
               & & & & 0b = RAW RX interrupt is disabled.\\
	\hline IRQ\_RAW\_TX & 0 & rw & 0x0 & Interrupt RAW TX queue.\\
               & & & & 1b = RAW TX interrupt is enabled.\\
               & & & & 0b = RAW TX interrupt is disabled.\\
\end{regdetails}

\subsubsection{PNDIRQ} \label{reg:pndirq}
\begin{regdescription}
	Name			\> PNDIRQ\\
	Relative Address	\> 0x00000068\\
	Width			\> 32 bits\\
	Access Type		\> ro\\
	Init Value		\> 0x00000000\\
	Description		\> Pending Interrupt Register. Used to report
	                           the cause of the interrupt\\
\end{regdescription}
\begin{regdetails}
	\hline reserved & 31:4 & ro & 0x0 & Reserved. Read data is zero.\\
	\hline IRQ\_RDMA\_RX & 3 & ro & 0x0 & Interrupt RDMA RX queue.\\
               & & & & This bit is set to 1b when the RDMA RX queue generates
               an interrupt.\\
               & & & & This bit is reset to 0b when the RDMA RX queue interrupt
               is acknowledged.\\
	\hline IRQ\_RDMA\_TX & 2 & ro & 0x0 & Interrupt RDMA TX queue.\\
               & & & & This bit is set to 1b when the RDMA TX queue generates
               an interrupt.\\
               & & & & This bit is reset to 0b when the RDMA TX queue interrupt
               is acknowledged.\\
	\hline IRQ\_RAW\_RX & 1 & ro & 0x0 & Interrupt RAW RX queue.\\
               & & & & This bit is set to 1b when the RAW RX queue generates
               an interrupt.\\
               & & & & This bit is reset to 0b when the RAW RX queue interrupt
               is acknowledged.\\
	\hline IRQ\_RAW\_TX & 0 & ro & 0x0 & Interrupt RAW TX queue.\\
               & & & & This bit is set to 1b when the RAW TX queue generates
               an interrupt.\\
               & & & & This bit is reset to 0b when the RAW TX queue interrupt
               is acknowledged.\\
\end{regdetails}



\section{Routing Registers} \label{mod:routing}
\begin{regdescription}
	Module Name 	\> Routing Registers\\
	Description 	\> Registers used to manage the routing table\\
\end{regdescription}

\subsection{Register summary}
\begin{regsummary}
    \hline \nameref{reg:routing_base} & 0x00000100+n & 8 & rw & 0x00 & Routing
    table base register\\
                            & n=0...255 & & & & \\
\end{regsummary}

\subsubsection{ROUTING\_BASE[n]} \label{reg:routing_base}
\begin{regdescription}
	Name			\> RAW\_RX\_BASE[n]\\
	Relative Address	\> 0x00000100+n, n=0...255\\
	Width (single row)	\> 8 bits\\
	Access Type		\> rw\\
	Init Value		\> 0x00000000\\
	Description		\> Routing table base register\\
	                        \> This array of registers represents the
	                        routing table.\\
	                        \> The array index represents the destination
	                        node id.\\
\end{regdescription}
\begin{regdetails}
	\hline reserved & 7:4 & rw & 0x0 & Reserved. Writes are ignored, read
	data is zero.\\
        \hline ENABLED\_IF\_3 & 3 & rw & 0x0 & When set to 1b, the node n (array
        index) is reachable through the interface 3.\\
        \hline ENABLED\_IF\_2 & 2 & rw & 0x0 & When set to 1b, the node n (array
        index) is reachable through the interface 2.\\
        \hline ENABLED\_IF\_1 & 1 & rw & 0x0 & When set to 1b, the node n (array
        index) is reachable through the interface 1.\\
        \hline ENABLED\_IF\_0 & 0 & rw & 0x0 & When set to 1b, the node n (array
        index) is reachable through the interface 0.\\
\end{regdetails}



\subsection{Queues Registers} \label{mod:queue}
\begin{regdescription}
	Module Name 	\> Queues Registers\\
	Description 	\> Registers used to manage the descriptor queues\\
	\textbf{Notes}  \> The offset of these registers probably will change\\
	                \> in the FORTH FPGA implementation\\
\end{regdescription}

\subsection{Register summary}
\begin{regsummary}
	\hline \nameref{reg:raw_tx_status} & 0x00000300 & 32 & ro & 0x0 & RAW TX
	queue status register\\
        \hline \nameref{reg:raw_tx_desc} & 0x00000400 & 1056 & wo & 0x0 & RAW TX
        queue descriptor register\\
	\hline \nameref{reg:raw_rx_status} & 0x00000310 & 32 & ro & 0x0 & RAW RX
	queue status register\\
        \hline \nameref{reg:raw_rx_desc} & 0x00000500 & 1056 & ro & 0x0 & RAW RX
        queue descriptor register\\
	\hline \nameref{reg:rdma_tx_status} & 0x00000320 & 32 & ro & 0x0 & RDMA
	TX queue status register\\
        \hline \nameref{reg:rdma_tx_desc} & 0x00000600 & 96 & wo & 0x0 & RDMA TX
        queue descriptor register\\
	\hline \nameref{reg:rdma_rx_status} & 0x00000330 & 32 & ro & 0x0 & RDMA
	RX queue status register\\
        \hline \nameref{reg:rdma_rx_desc} & 0x00000610 & 96 & ro & 0x0 & RDMA RX
        queue descriptor register\\
\end{regsummary}

\subsubsection{RAW\_TX\_STATUS} \label{reg:raw_tx_status}
\begin{regdescription}
	Name			\> RAW\_TX\_STATUS\\
	Relative Address	\> 0x00000300\\
	Width			\> 32 bits\\
	Access Type		\> ro\\
	Init Value		\> 0x00000000\\
	Description		\> RAW TX queue status register\\
\end{regdescription}
\begin{regdetails}
	\hline reserved & 31:1 & ro & 0x0 & Reserved. Writes are ignored, read
	data is zero.\\
	\hline QSTATUS\_AVAIL & 0 & ro & 0x0 & When set to 1b, in the RAW TX
	queue there are descriptors available\\
\end{regdetails}

\subsubsection{RAW\_TX\_DESC} \label{reg:raw_tx_desc}
\begin{regdescription}
	Name			\> RAW\_TX\_DESC\\
	Relative Address	\> 0x00000400\\
	Width (single row)	\> 1056 bits\\
	Access Type		\> wo\\
	Init Value		\> 0x00000000\\
	Description		\> RAW TX queue descriptor register\\
	                        \> These registers represent the descriptor of
	                        the RAW TX queue.\\
	                        \> The queue is handled internally by the
	                        hardware.\\
	                        \> The software can push new elements, writing
	                        these registers\\
\end{regdescription}
\begin{regdetails}
	\hline PAYLOAD & 1055:32 & ro & 0x0 & Payload of this message\\
	\hline reserved & 31:23 & wo & 0x0 & Reserved. Writes are ignored, read
	data is zero.\\
	\hline PAYLOAD\_SIZE & 22:16 & wo & 0x0 & Payload size of this message\\
	\hline DESTINATION & 15:8 & wo & 0x0 & Receiver node id or local
	interface to send message to the neighbour \\
	\hline reserved & 7:6 & wo & 0x0 & Reserved. Writes are ignored, read
	data is zero.\\
	\hline PORT & 5:3 & wo & 0x0 & Message port.\\
        \hline TYPE & 2:0 & wo & 0x0 & Message types.\\
                    & & & & (RAW\_DATA = 0, RAW\_NEIGHBOUR = 1, LONG\_DATA =
                    2, RDMA\_WRITE = 3, RDMA\_REQ = 4, RDMA\_RESPONSE = 5,
                    INIT = 6, ACK = 7)\\
\end{regdetails}

\subsubsection{RAW\_RX\_STATUS} \label{reg:raw_rx_status}
\begin{regdescription}
	Name			\> RAW\_RX\_STATUS\\
	Relative Address	\> 0x00000310\\
	Width			\> 32 bits\\
	Access Type		\> ro\\
	Init Value		\> 0x00000000\\
	Description		\> RAW RX queue status register\\
\end{regdescription}
\begin{regdetails}
	\hline reserved & 31:1 & wo & 0x0 & Reserved. Writes are ignored, read
	data is zero.\\
	\hline QSTATUS\_AVAIL & 0 & ro & 0x0 & When set to 1b, in the RAW RX
	queue there are descriptors available\\
\end{regdetails}

\subsubsection{RAW\_RX\_DESC} \label{reg:raw_rx_desc}
\begin{regdescription}
	Name			\> RAW\_RX\_DESC\\
	Relative Address	\> 0x00000400\\
	Width (single row)	\> 1056 bits\\
	Access Type		\> ro\\
	Init Value		\> 0x00000000\\
	Description		\> RAW RX queue descriptor register\\
	                        \> These registers represent the descriptor of
	                        the RAW RX queue.\\
	                        \> The queue is handled internally by the
	                        hardware.\\
	                        \> The software can pop new elements, reading
	                        these registers\\
\end{regdescription}
\begin{regdetails}
	\hline PAYLOAD & 1055:32 & ro & 0x0 & Payload of this message\\
	\hline reserved & 31:23 & ro & 0x0 & Reserved. Writes are ignored, read
	data is zero.\\
	\hline PAYLOAD\_SIZE & 22:16 & ro & 0x0 & Payload size of this message\\
	\hline SOURCE & 15:8 & ro & 0x0 & Sender node id or local interface
	where neighbour message is received\\
	\hline reserved & 7:6 & ro & 0x0 & Reserved. Writes are ignored, read
	data is zero.\\
	\hline PORT & 5:3 & ro & 0x0 & Message port.\\
        \hline TYPE & 2:0 & ro & 0x0 & Message types. \\
                    & & & & (RAW\_DATA = 0, RAW\_NEIGHBOUR = 1, LONG\_DATA =
                    2, RDMA\_WRITE = 3, RDMA\_REQ = 4, RDMA\_RESPONSE = 5,
                    INIT = 6, ACK = 7)\\
\end{regdetails}

\subsubsection{RDMA\_TX\_STATUS} \label{reg:rdma_tx_status}
\begin{regdescription}
	Name			\> RDMA\_TX\_STATUS\\
	Relative Address	\> 0x00000320\\
	Width			\> 32 bits\\
	Access Type		\> ro\\
	Init Value		\> 0x00000000\\
	Description		\> RDMA queue status register\\
\end{regdescription}
\begin{regdetails}
	\hline reserved & 31:1 & ro & 0x0 & Reserved. Writes are ignored, read
	data is zero.\\
	\hline QSTATUS\_AVAIL & 0 & ro & 0x0 & When set to 1b, in the RDMA RX
	queue there are descriptors available\\
\end{regdetails}

\subsubsection{RDMA\_TX\_DESC} \label{reg:rdma_tx_desc}
\begin{regdescription}
	Name			\> RDMA\_TX\_DESC\\
	Relative Address	\> 0x00000600\\
	Width (single row)	\> 96 bits\\
	Access Type		\> wo\\
	Init Value		\> 0x00000000\\
	Description		\> RDMA TX queue descriptor register\\
	                        \> These registers represent the descriptor of
	                        the RDMA TX queue.\\
	                        \> The queue is handled internally by the
	                        hardware.\\
	                        \> The software can push new elements, writing
	                        these registers\\
\end{regdescription}
\begin{regdetails}
        \hline DST\_ADDR & 95:64 & wo & 0x0 & Destination memory address \\
        \hline SRC\_ADDR & 63:32 & wo & 0x0 & Source memory address\\
	\hline reserved & 31:24 & wo & 0x0 & Reserved. Writes are ignored, read
	data is zero.\\
	\hline PAYLOAD\_SIZE & 23:16 & wo & 0x0 & Payload size of this RDMA\\
	\hline DESTINATION & 15:8 & wo & 0x0 & Receiver node id\\
	\hline reserved & 7:6 & wo & 0x0 & Reserved. Writes are ignored, read
	data is zero.\\
	\hline PORT & 5:3 & wo & 0x0 & RDMA port.\\
        \hline TYPE & 2:0 & wo & 0x0 & RDMA types.\\
                    & & & & (RAW\_DATA = 0, RAW\_NEIGHBOUR = 1, LONG\_DATA =
                    2, RDMA\_WRITE = 3, RDMA\_REQ = 4, RDMA\_RESPONSE = 5,
                    INIT = 6, ACK = 7)\\
\end{regdetails}

\subsubsection{RDMA\_RX\_STATUS} \label{reg:rdma_rx_status}
\begin{regdescription}
	Name			\> RDMA\_RX\_STATUS\\
	Relative Address	\> 0x00000330\\
	Width			\> 32 bits\\
	Access Type		\> ro\\
	Init Value		\> 0x00000000\\
	Description		\> RDMA RX queue status register\\
\end{regdescription}
\begin{regdetails}
	\hline reserved & 31:1 & wo & 0x0 & Reserved. Writes are ignored, read
	data is zero.\\
	\hline QSTATUS\_AVAIL & 0 & ro & 0x0 & When set to 1b, in the RDMA RX
	queue there are descriptors available\\
\end{regdetails}

\subsubsection{RDMA\_RX\_DESC} \label{reg:rdma_rx_desc}
\begin{regdescription}
	Name			\> RDMA\_RX\_DESC\\
	Relative Address	\> 0x00000610\\
	Width (single row)	\> 96 bits\\
	Access Type		\> ro\\
	Init Value		\> 0x00000000\\
	Description		\> RDMA RX queue descriptor register\\
	                        \> These registers represent the descriptor of
	                        the RAW RX queue.\\
	                        \> The queue is handled internally by the
	                        hardware.\\
	                        \> The software can pop new elements, reading
	                        these registers\\
\end{regdescription}
\begin{regdetails}
        \hline DST\_ADDR & 95:64 & ro & 0x0 & Destination memory address \\
        \hline SRC\_ADDR & 63:32 & ro & 0x0 & Source memory address\\
	\hline reserved & 31:24 & ro & 0x0 & Reserved. Writes are ignored, read
	data is zero.\\
	\hline PAYLOAD\_SIZE & 23:16 & ro & 0x0 & Payload size of this RDMA\\
	\hline SOURCE & 15:8 & ro & 0x0 & Sender node id\\
	\hline reserved & 7:6 & ro & 0x0 & Reserved. Writes are ignored, read
	data is zero.\\
	\hline PORT & 5:3 & ro & 0x0 & RDMA port.\\
        \hline TYPE & 2:0 & ro & 0x0 & RDMA types.\\
                    & & & & (RAW\_DATA = 0, RAW\_NEIGHBOUR = 1, LONG\_DATA =
                    2, RDMA\_WRITE = 3, RDMA\_REQ = 4, RDMA\_RESPONSE = 5,
                    INIT = 6, ACK = 7)\\
\end{regdetails}
\end{document}
