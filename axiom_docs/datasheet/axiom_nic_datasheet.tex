\documentclass[10pt,a4paper]{paper}
\usepackage[latin1]{inputenc}
\usepackage[margin=2cm]{geometry}
\usepackage{amsmath}
\usepackage{amsfonts}
\usepackage{amssymb}
\usepackage{graphicx}
\usepackage{tabularx}
\usepackage{environ}
\usepackage[colorlinks,linkcolor=blue]{hyperref}

\NewEnviron{regdescription}
	{\begin{tabbing}
	\hspace{150pt}\=\kill
		\BODY
	\end{tabbing}}

\NewEnviron{regglobalsummary}
	{\begin{tabularx}{\textwidth}{|l|l|l|l|l|X|}
	        \hline \textbf{Module} & \textbf{Register Name} & \textbf{Address} & \textbf{Width} & \textbf{Type} &  \textbf{Description} \\
		\BODY
	\hline
	\end{tabularx}}

\NewEnviron{regsummary}
	{\begin{tabularx}{\textwidth}{|l|l|l|l|l|X|}
	        \hline \textbf{Register Name} & \textbf{Address} & \textbf{Width} & \textbf{Type} & \textbf{Init Val} & \textbf{Description} \\
		\BODY
	\hline
	\end{tabularx}}

\NewEnviron{regdetails}
	{\begin{tabularx}{\textwidth}{|l|l|l|l|X|}
	        \hline \textbf{Field Name} & \textbf{Bit(s)} & \textbf{Type} & \textbf{Init Val} & \textbf{Description} \\
	 	\BODY
	\hline 
	\end{tabularx}}

\newcommand{\versionapi}{v0.3}

% Title Page
\title{Axiom NIC Datasheet}
\subtitle{\versionapi - \today}
\author{Evidence SRL}

\begin{document}
\maketitle


%\chapter{Axiom NIC}
\section{AXIOM NIC Registers}
\begin{regdescription}
	Total size      \> 6144 bytes\\
\end{regdescription}

\subsection{Register summary}
\begin{regglobalsummary}
        \hline \textbf{\nameref{mod:status}} & & & & & \\
                                             & & & & & \\
        \hline \nameref{mod:status} & \nameref{reg:version} & 0x0000 & 32 & ro & Version register\\
	\hline \nameref{mod:status} & \nameref{reg:status} & 0x0004 & 32 & ro & Status register\\
	\hline \nameref{mod:status} & \nameref{reg:ifnumber} & 0x0008 & 32 & ro & Number of interface\\
        \hline \nameref{mod:status} & \nameref{reg:ifinfo_base} & 0x0010+n, & 8 & ro & Interface status table\\
                               & n=0...3 & & & & \\
        \hline \textbf{\nameref{mod:control}} & & & & & \\
                                              & & & & & \\
	\hline \nameref{mod:control} & \nameref{reg:control} & 0x0040 & 32 & rw & Control register \\
	\hline \nameref{mod:control} & \nameref{reg:nodeid} & 0x0044 & 32 & rw & Node ID register \\
        \hline \textbf{\nameref{mod:interrupt}} & & & & & \\
                                                & & & & & \\
	\hline \nameref{mod:interrupt} & \nameref{reg:ackirq} & 0x0060 & 32 & wo & ACK Interrupt Register\\
	\hline \nameref{mod:interrupt} & \nameref{reg:mskirq} & 0x0064 & 32 & rw & Mask Interrupt Register\\
	\hline \nameref{mod:interrupt} & \nameref{reg:pndirq} & 0x0068 & 32 & ro & Pending Interrupt Register\\
        \hline \textbf{\nameref{mod:routing}} & & & & & \\
                                              & & & & & \\
        \hline \nameref{mod:routing} & \nameref{reg:routing_base} & 0x0100+n & 8 & rw & Routing table base register\\
                            & n=0...255 & & & & \\
        \hline \textbf{\nameref{mod:queue}} & & & & & \\
                                            & & & & & \\
        \hline \nameref{mod:queue} & \nameref{reg:small_tx_head} & 0x0300 & 32 & ro & SMALL TX queue head\\
	\hline \nameref{mod:queue} & \nameref{reg:small_tx_tail} & 0x0304 & 32 & ro & SMALL TX queue tail\\
	\hline \nameref{mod:queue} & \nameref{reg:small_tx_avail} & 0x0308 & 32 & ro & SMALL TX queue descriptors available\\
	\hline \nameref{mod:queue} & \nameref{reg:small_tx_push} & 0x030C & 32 & wo & SMALL TX queue push\\
        \hline \nameref{mod:queue} & \nameref{reg:small_tx_base} & 0x0800+8*n & 64 & wo & SMALL TX queue base register\\
                                & n=0...255 & & & & \\
	\hline \nameref{mod:queue} & \nameref{reg:small_rx_head} & 0x0310 & 32 & ro & SMALL TX queue head\\
	\hline \nameref{mod:queue} & \nameref{reg:small_rx_tail} & 0x0314 & 32 & ro & SMALL TX queue tail\\
	\hline \nameref{mod:queue} & \nameref{reg:small_rx_avail} & 0x0318 & 32 & ro & SMALL TX queue descriptors available\\
	\hline \nameref{mod:queue} & \nameref{reg:small_rx_pop} & 0x031C & 32 & wo & SMALL TX queue pop\\
        \hline \nameref{mod:queue} & \nameref{reg:small_rx_base} & 0x1000+8*n & 64 & ro & SMALL TX queue base register\\
                                & n=0...255 & & & & \\
\end{regglobalsummary}


\section{Status Registers} \label{mod:status}
\begin{regdescription}
	Module Name 	\> Status Registers\\
	Description 	\> Registers used to check the status of the device\\
\end{regdescription}

\subsection{Register summary}
\begin{regsummary}
	\hline \nameref{reg:version} & 0x0000 & 32 & ro & 0x0 & Version register\\
	\hline \nameref{reg:status} & 0x0004 & 32 & ro & 0x0 & Status register\\
	\hline \nameref{reg:ifnumber} & 0x0008 & 32 & ro & 0x0 & Number of interface\\
        \hline \nameref{reg:ifinfo_base} & 0x0010+n, & 8 & ro & 0x0 & Interface status table\\
                               & n=0...7 & & & & \\
\end{regsummary}


\subsubsection{VERSION} \label{reg:version}
\begin{regdescription}
	Name			\> VERSION\\
	Relative Address	\> 0x00000000\\
	Width			\> 32 bits\\
	Access Type		\> ro\\
	Init Value		\> 0x00000000\\
	Description		\> Version register\\
\end{regdescription}
\begin{regdetails}
	\hline reserved & 31:16 & ro & 0x0 & Reserved. Writes are ignored, read data is zero.\\
	\hline BITSTREAM & 15:8 & ro & 0x0 & Bitstream version\\
	\hline BOARD & 7:0 & ro & 0x0 & Axiom board model\\
\end{regdetails}


\subsubsection{STATUS} \label{reg:status}
\begin{regdescription}
	Name			\> STATUS\\
	Relative Address	\> 0x00000004\\
	Width			\> 32 bits\\
	Access Type		\> ro\\
	Init Value		\> 0x00000000\\
	Description		\> Status register\\
\end{regdescription}
\begin{regdetails}
	\hline reserved & 31:1 & ro & 0x0 & Reserved. Writes are ignored, read data is zero.\\
	\hline TBD & 0 & rw & 0x0 & TBD\\
\end{regdetails}


\subsubsection{IFNUMBER} \label{reg:ifnumber}
\begin{regdescription}
	Name			\> IFNUMBER\\
	Relative Address	\> 0x00000008\\
	Width			\> 32 bits\\
	Access Type		\> ro\\
	Init Value		\> 0x00\\
	Description		\> Number of interface\\
\end{regdescription}
\begin{regdetails}
	\hline reserved & 31:3 & ro & 0x0 & Reserved. Writes are ignored, read data is zero.\\
	\hline IFNUMBER & 2:0 & ro & 0x2 & Number of interface which are present on a node.\\
\end{regdetails}


\subsubsection{IFINFO\_BASE[n]} \label{reg:ifinfo_base}
\begin{regdescription}
	Name			\> IFINFO\_BASE[n]\\
	Relative Address	\> 0x00000010+n, n=0...7\\
	Width (single row)	\> 8 bits\\
	Access Type		\> ro\\
	Init Value		\> 0x00\\
	Description		\> Interface status table\\
\end{regdescription}
\begin{regdetails}
	\hline reserved & 7:3 & ro & 0x0 & Reserved. Writes are ignored, read data is zero.\\
	\hline CONNECTED & 2 & ro & 0x0 & Interface connected status.\\
               & & & & 1b = interface is physically connected to another board.\\
               & & & & 0b = interface is disconnected.\\
	\hline RX & 1 & ro & 0x0 & Interface RX functionality.\\
               & & & & This bit is set when the interface can work as RX.\\
	\hline TX & 0 & ro & 0x0 & Interface TX functionality.\\
               & & & & This bit is set when the interface can work as TX.\\
\end{regdetails}



\section{Control Registers} \label{mod:control}
\begin{regdescription}
	Module Name 	\> Control Registers\\
	Description 	\> Registers used to control the device\\
\end{regdescription}

\subsection{Register summary}
\begin{regsummary}
	\hline \nameref{reg:control} & 0x0040 & 32 & rw & 0x0 & Control register \\
	\hline \nameref{reg:nodeid} & 0x0044 & 32 & rw & 0x0 & Node ID register \\
\end{regsummary}


\subsubsection{CONTROL} \label{reg:control}
\begin{regdescription}
	Name			\> CONTROL\\
	Relative Address	\> 0x00000040\\
	Width			\> 32 bits\\
	Access Type		\> rw\\
	Init Value		\> 0x00000000\\
	Description		\> Control register. Used to enable local transmission ACKs\\
	                        \> and/or the PHY loopback mode configuration.\\
\end{regdescription}
\begin{regdetails}
	\hline reserved & 31:1 & rw & 0x0 & Reserved. Writes are ignored, read data is zero.\\
	\hline LOOPBACK & 0 & rw & 0x0 & When set to 1b, all the interfaces are in loopback mode.\\
\end{regdetails}


\subsubsection{NODEID} \label{reg:nodeid}
\begin{regdescription}
	Name			\> NODEID\\
	Relative Address	\> 0x00000044\\
	Width			\> 32 bits\\
	Access Type		\> rw\\
	Init Value		\> 0x00\\
	Description		\> Node ID register\\
\end{regdescription}
\begin{regdetails}
	\hline NODEID & 7:0 & rw & 0x0 & Defines the ID of a local node.\\
\end{regdetails}



\section{Interrupt Registers} \label{mod:interrupt}
\begin{regdescription}
	Module Name 	\> Interrupt Registers\\
	Description 	\> Registers used to handle interrupts\\
\end{regdescription}

\subsection{Register summary}
\begin{regsummary}
	\hline \nameref{reg:ackirq} & 0x0060 & 32 & wo & 0x0 & ACK Interrupt Register\\
	\hline \nameref{reg:mskirq} & 0x0064 & 32 & rw & 0x0 & Mask Interrupt Register\\
	\hline \nameref{reg:pndirq} & 0x0068 & 32 & ro & 0x0 & Pending Interrupt Register\\
\end{regsummary}

\subsubsection{ACKIRQ} \label{reg:ackirq}
\begin{regdescription}
	Name			\> ACKIRQ\\
	Relative Address	\> 0x00000060\\
	Width			\> 32 bits\\
	Access Type		\> wo\\
	Init Value		\> 0x00000000\\
	Description		\> ACK Interrupt Register. Used to send an ack for the interrupts\\
\end{regdescription}
\begin{regdetails}
	\hline reserved & 31:2 & wo & 0x0 & Reserved. Writes are ignored.\\
	\hline IRQ\_SMALL\_TX & 1 & wo & 0x0 & Interrupt SMALL TX queue.\\
               & & & & When set to 1b, the SMALL TX interrupt is acknowledged.\\
	\hline IRQ\_SMALL\_RX & 0 & wo & 0x0 & Interrupt SMALL RX queue.\\
               & & & & When set to 1b, the SMALL RX interrupt is acknowledged.\\
\end{regdetails}

\subsubsection{MSKIRQ} \label{reg:mskirq}
\begin{regdescription}
	Name			\> MSKIRQ\\
	Relative Address	\> 0x00000064\\
	Width			\> 32 bits\\
	Access Type		\> rw\\
	Init Value		\> 0x00000000\\
	Description		\> Mask Interrupt Register. Used to enable/disable interrupts\\
\end{regdescription}
\begin{regdetails}
	\hline reserved & 31:2 & rw & 0x0 & Reserved. Writes are ignored, read data is zero.\\
	\hline IRQ\_SMALL\_TX & 1 & rw & 0x0 & Interrupt SMALL TX queue.\\
               & & & & 1b = SMALL TX interrupt is enabled.\\
               & & & & 0b = SMALL TX interrupt is disabled.\\
	\hline IRQ\_SMALL\_RX & 0 & rw & 0x0 & Interrupt SMALL RX queue.\\
               & & & & 1b = SMALL RX interrupt is enabled.\\
               & & & & 0b = SMALL RX interrupt is disabled.\\
\end{regdetails}

\subsubsection{PNDIRQ} \label{reg:pndirq}
\begin{regdescription}
	Name			\> PNDIRQ\\
	Relative Address	\> 0x00000068\\
	Width			\> 32 bits\\
	Access Type		\> ro\\
	Init Value		\> 0x00000000\\
	Description		\> Pending Interrupt Register. Used to report the cause of the interrupt\\
\end{regdescription}
\begin{regdetails}
	\hline reserved & 31:2 & ro & 0x0 & Reserved. Read data is zero.\\
	\hline IRQ\_SMALL\_TX & 1 & ro & 0x0 & Interrupt SMALL TX queue.\\
               & & & & This bit is set to 1b when the SMALL TX queue generates an interrupt.\\
               & & & & This bit is reset to 0b when the SMALL TX queue interrupt is acknowledged.\\
	\hline IRQ\_SMALL\_RX & 0 & ro & 0x0 & Interrupt SMALL RX queue.\\
               & & & & This bit is set to 1b when the SMALL RX queue generates an interrupt.\\
               & & & & This bit is reset to 0b when the SMALL RX queue interrupt is acknowledged.\\
\end{regdetails}



\section{Routing Registers} \label{mod:routing}
\begin{regdescription}
	Module Name 	\> Routing Registers\\
	Description 	\> Registers used to manage the routing table\\
\end{regdescription}

\subsection{Register summary}
\begin{regsummary}
    \hline \nameref{reg:routing_base} & 0x0100+n & 8 & rw & 0x00 & Routing table base register\\
                            & n=0...255 & & & & \\
\end{regsummary}

\subsubsection{ROUTING\_BASE[n]} \label{reg:routing_base}
\begin{regdescription}
	Name			\> SMALL\_RX\_BASE[n]\\
	Relative Address	\> 0x00000100+n, n=0...255\\
	Width (single row)	\> 8 bits\\
	Access Type		\> rw\\
	Init Value		\> 0x00\\
	Description		\> Routing table base register\\
	                        \> This array of registers represents the routing table.\\
	                        \> The array index represents the destination node id.\\
\end{regdescription}
\begin{regdetails}
	\hline reserved & 7:4 & rw & 0x0 & Reserved. Writes are ignored, read data is zero.\\
        \hline ENABLED\_IF\_3 & 3 & rw & 0x0 & When set to 1b, the node n (array index) is reachable through the interface 3.\\
        \hline ENABLED\_IF\_2 & 2 & rw & 0x0 & When set to 1b, the node n (array index) is reachable through the interface 2.\\
        \hline ENABLED\_IF\_1 & 1 & rw & 0x0 & When set to 1b, the node n (array index) is reachable through the interface 1.\\
        \hline ENABLED\_IF\_0 & 0 & rw & 0x0 & When set to 1b, the node n (array index) is reachable through the interface 0.\\
\end{regdetails}



\subsection{Queues Registers} \label{mod:queue}
\begin{regdescription}
	Module Name 	\> Queues Registers\\
	Description 	\> Registers used to manage the descriptor queues\\
	\textbf{Notes}  \> The offset of these registers probably will change\\
	                \> in the FORTH FPGA implementation\\
\end{regdescription}

\subsection{Register summary}
\begin{regsummary}
        \hline \nameref{reg:small_tx_head} & 0x0300 & 32 & ro & 0x0 & SMALL TX queue head\\
	\hline \nameref{reg:small_tx_tail} & 0x0304 & 32 & ro & 0x0 & SMALL TX queue tail\\
	\hline \nameref{reg:small_tx_avail} & 0x0308 & 32 & ro & 0x0 & SMALL TX queue descriptors available\\
	\hline \nameref{reg:small_tx_push} & 0x030C & 32 & wo & 0x0 & SMALL TX queue push\\
        \hline \nameref{reg:small_tx_base} & 0x0800+8*n & 64 & wo & 0x0 & SMALL TX queue base register\\
                                & n=0...255 & & & & \\
	\hline \nameref{reg:small_rx_head} & 0x0310 & 32 & ro & 0x0 & SMALL TX queue head\\
	\hline \nameref{reg:small_rx_tail} & 0x0314 & 32 & ro & 0x0 & SMALL TX queue tail\\
	\hline \nameref{reg:small_rx_avail} & 0x0318 & 32 & ro & 0x0 & SMALL TX queue descriptors available\\
	\hline \nameref{reg:small_rx_pop}  & 0x031C & 32 & wo & 0x0 & SMALL TX queue pop\\
        \hline \nameref{reg:small_rx_base} & 0x1000+8*n & 64 & ro & 0x0 & SMALL TX queue base register\\
                                & n=0...255 & & & & \\
\end{regsummary}

\subsubsection{SMALL\_TX\_HEAD} \label{reg:small_tx_head}
\begin{regdescription}
	Name			\> SMALL\_TX\_HEAD\\
	Relative Address	\> 0x00000300\\
	Width			\> 32 bits\\
	Access Type		\> ro\\
	Init Value		\> 0x00\\
	Description		\> SMALL TX queue head\\
\end{regdescription}
\begin{regdetails}
	\hline reserved & 31:8 & ro & 0x0 & Reserved. Writes are ignored, read data is zero.\\
	\hline SMALL\_TX\_HEAD & 7:0 & ro & 0x0 & SMALL TX queue head.\\
               & & & &  This register contains the head pointer for the SMALL TX queue.
                        The head (in the TX queue) points to the first descriptor to be sent by
                        the network card. Hardware controls this pointer. When the descriptor
                        pointed by the head is sent, this register is incremented.\\
\end{regdetails}

\subsubsection{SMALL\_TX\_TAIL} \label{reg:small_tx_tail}
\begin{regdescription}
	Name			\> SMALL\_TX\_TAIL\\
	Relative Address	\> 0x00000304\\
	Width			\> 32 bits\\
	Access Type		\> ro\\
	Init Value		\> 0x00\\
	Description		\> SMALL TX queue tail\\
\end{regdescription}
\begin{regdetails}
	\hline reserved & 31:8 & ro & 0x0 & Reserved. Writes are ignored, read data is zero.\\
	\hline SMALL\_TX\_TAIL & 7:0 & ro & 0x0 & SMALL TX queue tail.\\
               & & & &  This register contains the tail pointer for the SMALL TX queue.
                        The tail (in the TX queue) points to the first empty descriptor
                        which the software can fill. Hardware controls this pointer. When the
                        software writes on \nameref{reg:small_tx_push}, this register
                        is incremented.\\
\end{regdetails}

\subsubsection{SMALL\_TX\_AVAIL} \label{reg:small_tx_avail}
\begin{regdescription}
	Name			\> SMALL\_TX\_AVAIL\\
	Relative Address	\> 0x00000308\\
	Width			\> 32 bits\\
	Access Type		\> ro\\
	Init Value		\> 0x00\\
	Description		\> SMALL TX queue descriptors available\\
\end{regdescription}
\begin{regdetails}
	\hline reserved & 31:16 & ro & 0x0 & Reserved. Writes are ignored, read data is zero.\\
	\hline SMALL\_TX\_AVAIL & 15:0 & ro & 0x0 & SMALL TX queue descriptors available\\
               & & & &  This register contains the number of free descriptor
                        available in the SMALL TX queue, which the software can fill.\\
\end{regdetails}

\subsubsection{SMALL\_TX\_PUSH} \label{reg:small_tx_push}
\begin{regdescription}
	Name			\> SMALL\_TX\_PUSH\\
	Relative Address	\> 0x0000030C\\
	Width			\> 32 bits\\
	Access Type		\> wo\\
	Init Value		\> 0x00\\
	Description		\> SMALL TX queue push\\
\end{regdescription}
\begin{regdetails}
	\hline reserved & 31:16 & wo & 0x0 & Reserved. Writes are ignored, read data is zero.\\
	\hline SMALL\_TX\_PUSH & 15:0 & wo & 0x0 & SMALL TX queue push.\\
                       & & & & When set to value X, the \nameref{reg:small_tx_tail} is
                                incremented by X (the hardware prevents the head overtaking).
                                After that, the new X descriptors are ready
                                to be sent by the hardware. The software should fill the
                                descriptors pointed by \nameref{reg:small_tx_tail},
                                before setting this registers.\\
\end{regdetails}

\subsubsection{SMALL\_TX\_BASE[n]} \label{reg:small_tx_base}
\begin{regdescription}
	Name			\> SMALL\_TX\_BASE[n]\\
	Relative Address	\> 0x00000800+8*n, n=0...255\\
	Width (single row)	\> 64 bits\\
	Access Type		\> wo\\
	Init Value		\> 0x00\\
	Description		\> SMALL TX queue base register\\
	                        \> This array of registers represents the queue of SMALL TX descriptors.\\
	                        \> Each descriptor is 64-bit.\\
\end{regdescription}
\begin{regdetails}
	\hline PAYLOAD & 63:32 & wo & 0x0 & Data to be sent. \\
	\hline reserved & 31:16 & wo & 0x0 & Reserved. Writes are ignored, read data is zero.\\
	\hline DESTINATION & 15:8 & wo & 0x0 & Receiver node id or local interface to send message to the neighbour \\
	\hline reserved & 7:6 & wo & 0x0 & Reserved. Writes are ignored, read data is zero.\\
	\hline PORT & 5:3 & wo & 0x0 & SMALL message port.\\
        \hline FLAGS: TBD & 2:1 & wo & 0x0 & SMALL message flags. TBD\\
        \hline FLAGS: NEIGHBOUR& 0 & wo & 0x0 & SMALL message flags: NEIGHBOUR. When set to 1b, the message will go to the neighbor node\\
\end{regdetails}

\subsubsection{SMALL\_RX\_HEAD} \label{reg:small_rx_head}
\begin{regdescription}
	Name			\> SMALL\_RX\_HEAD\\
	Relative Address	\> 0x00000310\\
	Width			\> 32 bits\\
	Access Type		\> ro\\
	Init Value		\> 0x00\\
	Description		\> SMALL RX queue head\\
\end{regdescription}
\begin{regdetails}
	\hline reserved & 31:8 & ro & 0x0 & Reserved. Writes are ignored, read data is zero.\\
	\hline SMALL\_RX\_HEAD & 7:0 & ro & 0x0 & SMALL RX queue head.\\
               & & & &  This register contains the head pointer for the SMALL RX queue.
                        The head (in the RX queue) points to the first descriptor received by
                        the network card, which the software can read.
                        Hardware controls this pointer. When the software writes on
                        \nameref{reg:small_rx_pop}, this register is incremented.\\
\end{regdetails}

\subsubsection{SMALL\_RX\_TAIL} \label{reg:small_rx_tail}
\begin{regdescription}
	Name			\> SMALL\_RX\_TAIL\\
	Relative Address	\> 0x00000314\\
	Width			\> 32 bits\\
	Access Type		\> ro\\
	Init Value		\> 0x00\\
	Description		\> SMALL RX queue tail\\
\end{regdescription}
\begin{regdetails}
	\hline reserved & 31:8 & ro & 0x0 & Reserved. Writes are ignored, read data is zero.\\
	\hline SMALL\_RX\_TAIL & 7:0 & ro & 0x0 & SMALL RX queue tail.\\
               & & & &  This register contains the tail pointer for the SMALL RX queue.
                        The tail (in the RX queue) points to the first empty descriptor
                        which the hardware can fill. Hardware controls this pointer.
                        When the descriptor pointed by the tail is filled by the hardware,
                        this register is incremented.\\
\end{regdetails}

\subsubsection{SMALL\_RX\_AVAIL} \label{reg:small_rx_avail}
\begin{regdescription}
	Name			\> SMALL\_RX\_AVAIL\\
	Relative Address	\> 0x00000318\\
	Width			\> 32 bits\\
	Access Type		\> ro\\
	Init Value		\> 0x00\\
	Description		\> SMALL RX queue descriptors available\\
\end{regdescription}
\begin{regdetails}
	\hline reserved & 31:16 & wo & 0x0 & Reserved. Writes are ignored, read data is zero.\\
	\hline SMALL\_RX\_AVAIL & 15:0 & ro & 0x0 & SMALL RX queue descriptors available\\
               & & & &  This register contains the number of filled descriptor
                        available in the SMALL RX queue, which the software can read.\\
\end{regdetails}

\subsubsection{SMALL\_RX\_POP} \label{reg:small_rx_pop}
\begin{regdescription}
	Name			\> SMALL\_RX\_POP\\
	Relative Address	\> 0x0000031C\\
	Width			\> 32 bits\\
	Access Type		\> wo\\
	Init Value		\> 0x00\\
	Description		\> SMALL RX queue pop\\
\end{regdescription}
\begin{regdetails}
	\hline reserved & 31:16 & wo & 0x0 & Reserved. Writes are ignored, read data is zero.\\
	\hline SMALL\_RX\_POP & 15:0 & wo & 0x0 & SMALL RX queue pop.\\
                       & & & & When set to value X, the \nameref{reg:small_rx_head} is
                                incremented by X (the hardware prevents the tail overtaking).
                                After that, the new X descriptors are consumed by the
                                software and ready to be reused by the hardware.
                                The software should read the
                                descriptors pointed by \nameref{reg:small_rx_head},
                                before setting this register.\\
\end{regdetails}

\subsubsection{SMALL\_RX\_BASE[n]} \label{reg:small_rx_base}
\begin{regdescription}
	Name			\> SMALL\_RX\_BASE[n]\\
	Relative Address	\> 0x00001000+8*n, n=0...255\\
	Width (single row)	\> 64 bits\\
	Access Type		\> ro\\
	Init Value		\> 0x00\\
	Description		\> SMALL RX queue base register\\
	                        \> This array of registers represents the queue of SMALL RX descriptors.\\
	                        \> Each descriptor is 64-bit.\\
\end{regdescription}
\begin{regdetails}
	\hline PAYLOAD & 63:32 & ro & 0x0 & Data received. \\
	\hline reserved & 31:16 & ro & 0x0 & Reserved. Writes are ignored, read data is zero.\\
	\hline SOURCE & 15:8 & ro & 0x0 & Sender node id or local interface where neighbour message is received\\
	\hline reserved & 7:6 & ro & 0x0 & Reserved. Writes are ignored, read data is zero.\\
	\hline PORT & 5:3 & ro & 0x0 & SMALL message port.\\
        \hline FLAGS: TBD & 2:1 & ro & 0x0 & SMALL message flags. (NEIGHBOUR, etc.)\\
        \hline FLAGS: NEIGHBOUR& 0 & ro & 0x0 & SMALL message flags: NEIGHBOUR. When set to 1b, the message comes from the neighbor node\\
\end{regdetails}

\end{document}
